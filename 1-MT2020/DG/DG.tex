\documentclass[a4paper,11pt]{article}

\usepackage{../../zyancamlec}

\def\ntripos{Mathematical Tripos}
\def\npart{III}
\def\ncourse{Differential Geometry}
\def\nscourse{DiffGeo}
\def\nlecturer{J.\ Smith}
\def\nterm{Michaelmas}
\def\nyear{2020}

\begin{document}
	\maketitlepage
	\preliminaries

	\section*{Introduction}

	Differential geometry is the study of manifolds --- spaces built from smoothly gluing together open sets in Euclidean space —-- and structures that live on or in them. The goal of this course is to introduce the main ideas on both the abstract conceptual (‘coordinate-free’) level and the concrete computational (‘in coordinates’) level, and to develop fluency in passing between them. This will lay the foundation for future study in geometry and topology, and provide the language for modern theoretical physics. Throughout the emphasis will be on building up geometric intuition. Topics will include:


	\begin{itemize}
		\item Manifolds, tangent and cotangent spaces, smooth maps and their derivatives. Tangent and cotangent bundles, tensors. Vector fields, flows, the Lie derivative.
		\item Differential forms, the exterior derivative, de Rham cohomology. Orientability. Integration and Stokes’s theorem. Frobenius integrability.
		\item Lie groups and algebras. Principal bundles, connections (from multiple perspectives), curvature. Associated bundles, reduction of the structure group, vector bundles.
		\item Riemannian metrics, the Levi-Civita connection, geodesics and the exponential map. The Riemann tensor and its symmetries and contractions. The Hodge star, the Laplacian, statement of the Hodge decomposition.
	\end{itemize}

	
	\section*{Pre-requisites}
	
	Familiarity with point set topology (including compactness), multi-variable calculus (including the inverse function theorem), and linear algebra (including dual spaces and bilinear forms) is essential. No previous exposure to geometry will be assumed.

	\newpage
	\tableofcontents
	\newpage
	\maintext
	Hi
	
\end{document}