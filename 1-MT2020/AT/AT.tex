\documentclass[a4paper,11pt]{article}

\usepackage{../../zyancamlec}

\def\ntripos{Mathematical Tripos}
\def\npart{III}
\def\ncourse{Algebraic Topology}
\def\nscourse{AlgTop}
\def\nlecturer{I.\ Smith}
\def\nterm{Michaelmas}
\def\nyear{2020}

\begin{document}
	\maketitlepage
	\preliminaries

	\section*{Course Information}

	Algebraic Topology permeates modern pure mathematics and theoretical physics. This course will focus on (co)homology, with an emphasis on applications to the topology of manifolds. We will cover singular and cellular (co)homology; degrees of maps and cup-products; cohomology with compact supports and Poincar\'e duality; and Thom isomorphism and the Euler class. The course will not specifically assume any knowledge of algebraic topology, but will go quite fast in order to reach more interesting material, so some previous exposure to chain complexes (e.g. simplicial homology) would certainly be helpful.
	
	\section*{Pre-requisites}

	Basic topology: topological spaces, compactness and connectedness, at the level of Sutherland’s book. Some knowledge of the fundamental group would be helpful though not a requirement. Hatcher’s book and Bott and Tu’s book are especially recommended for accompanying the course, but there are many other suitable texts.

	\newpage
	\tableofcontents
	\newpage
	\maintext
	\setcounter{section}{-1}
	\section{Introduction} \lecnr{1}
	
	\emph{Algebraic Topology} concerns the \emph{connectivity} properties of topological spaces.

	\begin{defi}
		A space $X$ is \emph{path-connected} if for $p,q\in X$, $\exists \gamma: [0,1] \to X$ continuous with $\gamma(0) = p, \gamma(1) = q$. 
	\end{defi}

	\needfig{1}

	\begin{ex}
		$\mathbb{R}$ is path-connected; $\mathbb{R} \backslash \{0\}$ is not.
	\end{ex}

	\begin{cor}[The intermediate value theorem]
		If $f : \mathbb{R} \to \mathbb{R}$ is continuous and $x < y$ satisfy $f(x)<0, f(y)>0$ then $f$ takes the value 0 on $[x,y]$.
	\end{cor}

	\begin{proof}
		Otherwise, $f^{-1}(-\infty,0)\cup f^{-1}(0,\infty)$ disconnects $[x,y]$, \#. 
	\end{proof}

	\begin{defi}
		Let $X,Y$ be topological spaces. We say maps $f_0, f_1 : Y \to X$ are \emph{homotopic} if $\exists F : Y \times [0,1] \to X$ continuous such that
		\[
			F|_{Y \times \{0\}} = f_0, \qquad F|_{Y \times \{1\}} = f_1
		\]
		
		We write $f_0 \simeq f_1$ (or $f_0 \underset{F}{\simeq} f_1)$.
	\end{defi}
	\needfig{2}

	\begin{exer}
		(On example sheet 1) $\simeq$ is an equivalence relation on the set of maps from $Y$ to $X$.
	\end{exer}

	\begin{nt}
		$X$ is \emph{path-connected} iff every two maps $\{\text{point}\} \to X$ are homotopic.
	\end{nt}

	\begin{defi}
		$X$ is \emph{simply-connected} if every two maps $S^1 \to X$ are homotopic.
	\end{defi}

	\begin{nt}
		We often denote
		\[
			S^1 = \{z \in \mathbb{C} : \abs{z} = 1\}, \qquad S^n = \{x\in \mathbb{R}^{n+1} : \norm{x} = 1\}
		\]
	\end{nt}

	\begin{ex}
		$\mathbb{R}^2$ is simply connected; $\mathbb{R}^2 \backslash \{0\}$ is not.

		From complex analysis we know $\gamma: S^1 \to \mathbb{R}^2 \backslash \{0\}$ has a \emph{winding number} or \emph{degree} $\deg (\gamma) \in \mathbb{Z}$, for which
		\begin{enumerate}
			\item If $\gamma_n (t) = e^{2\pi \mathrm{i} n t}$ then $\deg (\gamma_n) = n$;
			\item $\deg(\gamma_1) = \deg(\gamma_2)$ if $\gamma_1 \simeq \gamma_2$.  
		\end{enumerate} 
		\needfig{3}

		For \emph{differentiable} $\gamma$, 
		$$\deg(\gamma) = \int_\gamma \frac{\dd{z}}{z}.$$
	\end{ex}
	
	\begin{cor}[Fundamental theorem of algebra]
		Every non-constant complex polynomial has a root.
	\end{cor}

	\begin{proof}
		Let $f(z) = z^n + a_1 z^{n-1} + \cdots + a_n$ be non-constant and WLOG monic. Suppose $f(z) \neq 0, \forall z \in \mathbb{C}$, let $\gamma_R (t):= f\left( R e^{2\pi \mathrm{i} t} \right)$ so that $\gamma_R : S^1 \to \mathbb{R}^2 \backslash \{0\}$. We know that
		\[
			\gamma_0 \text{ is constant} \quad \Rightarrow\quad \deg(\gamma_0) = 0 \quad \Rightarrow\quad \deg(\gamma_R) = 0, \quad \forall R
		\]
		However, if we take $R \gg \sum_i \abs{a_i}$, let $f_s(z) = z^n + s \left( a_1 z^{n-1} + \cdots + a_n \right)$ with $0 \leq s \leq 1$. On the circle $\abs{z} = R$, $f_s(z) \neq 0, \forall s$.

		Therefore, if $\gamma_{R,s}(t) := f_s\left( R e^{2\pi \mathrm{i} t} \right)$ then we have $\gamma_{R,1} = \gamma_R$ and $\gamma_{R,0} : t \mapsto R^n e^{2\pi \mathrm{i} n t}$.
		
		Clearly, we have
		\[
			\deg(\gamma_{R,1}) = 0 \neq n = \deg(\gamma_{R,0}) 
		\]
		as non-constant property suggests $n \neq 0$. This is a \#.
	\end{proof}

	\begin{defi}
		$X$ is \emph{$k$-connected} if every two maps $S^i \to X$ are homotopic whenever $i \leq k$.
	\end{defi}

	\begin{ex}
		$\mathbb{R}^n$ is $(n-1)$-connected; $\mathbb{R}^n \backslash \{0\}$ is not. Maps $S^{n-1} \to \mathbb{R}^n \backslash \{0\}$ have a homotopy-invariant degree $\in \mathbb{Z}$ and deg(inclusion) = 1, deg(constant) = 0. (We'll prove it later.)
	\end{ex}

	\begin{cor}[Brouwer's theorem]
		For closed unit ball $\bar{B}^n = \{x\in \mathbb{R}^n : \norm{x}\leq 1\}$, any map $f : \bar{B}^n \to \bar{B}^n$ has a fixed point.
	\end{cor}

	\begin{proof}
		Suppose $f$ has no fixed point. Let $\gamma_R (v) := Rv - f(Rv)$ where $0 \leq R \leq 1$ and $v \in S^{n-1} = \partial \bar{B}^n$. Our assumption suggests $\gamma_R$ takes values in $\mathbb{R}^n \backslash \{0\}$.

		According to homotopy invariance, as $\gamma_0$ is constant, we have $\deg(\gamma_0) = 0$ hence $\deg(\gamma_1) = 0$.
		
		Let $\gamma_{1,s}(v) := v - s f(v)$ for $0 \leq s \leq 1$. Then $\gamma_{1,1} = \gamma_1$ and $\text{image}(\gamma_{1,s}) \subseteq \mathbb{R} \backslash \{0\}$ as $\norm{v} = 1, \norm{s f(v)} = \abs{s}\norm{f(v)} < 1$ if $s < 1$.
		
		Therefore, we have $\deg(\gamma_{1,0}) = \deg(\gamma_{1,1}) = 0$ by homotopy invariance. However, the inclusion $\gamma_{1,0}$ should have degree 1, thus \#.
	\end{proof}

	\begin{defi}
		$f: X \to Y$ is a \emph{homotopy-equivalence} if $\exists g : Y \to X$ such that $f \cpf g \simeq \id_Y, g \cpf f \simeq \id_X$. (We call $g$ a ``homotopy inverse'' for $f$.)
	\end{defi}

	\begin{nt}
		The homotopy equivalence can be shown as an equivalence relation on spaces.
	\end{nt}

	\begin{ex}
		If $X,Y$ are homeomorphic they are trivially homotopy equivalent: simply by taking $g = f^{-1}$.
	\end{ex}

	\begin{ex}
		$\mathbb{R}^n \backslash \{0\} \simeq S^{n-1}$. 
		
		Let
		\[
			f: \mathbb{R}^n \backslash \{0\} \to S^{n-1}, \quad v \xmapsto{f} \frac{v}{\norm{v}}
		\]
		\[
			g: S^{n-1} \hookrightarrow \mathbb{R}^n \backslash \{0\} \text{ by inclusion}
		\]
		Then
		\[
			f \cpf g = \id_{S^{n-1}}, \qquad g \cpf f \underset{F}{\simeq} \id_{\mathbb{R}^n \backslash \{0\}}
		\]
		via the homotopy
		\[
			F(t,v) = tv + (1-t) \frac{v}{\norm{v}}
		\]
		\needfig{4}
	\end{ex}

	\begin{ex}
		$\{0\} \xhookrightarrow{\sim} \mathbb{R}^n$ is a homotopy equivalence. (Check!)
	\end{ex}

	\begin{defi}
		If a space $X \simeq \{\text{point}\}$ we say $X$ is \emph{contractible}. 
	\end{defi}

	Talking about all these, we emphasise that
	\[
		\boxed{\text{Algebraic topology is the study of topological spaces up to homotopy equivalence.}}
	\]
	
	The main idea is that: homeomorphism is too delicate as a relation, but homotopy equivalence keeps track of ``essential'' topological information. More precisely, we assign
	\begin{align*}
		\{\text{Spaces}\} & \to \{\text{Groups}\}\\
		\{\text{Maps of spaces}\} & \to \{\text{Homomorphisms of groups}\}
	\end{align*}
	so we get algebraic invariants. (They are defined for \emph{all} spaces, but have more structure and use/interest for ``nicer'' spaces.)

	The classical first attempt of algebraic topology would be \emph{homotopy theory}. One can \emph{concatenate} loops:
	\needfig{5}
	for
	\[
		\gamma * \tau (t) = \begin{cases}
			\gamma(2t),& t \leq \frac{1}{2}\\
			\tau(1 - 2t),& t \geq \frac{1}{2}
		\end{cases}
	\]
	which leads to
	\[
		\{\text{Maps }S^1 \xrightarrow{\gamma} X\} / \simeq \quad  \longrightarrow \quad \pi_1(X,x_0)
	\]
	where $\gamma$ fixes $\gamma(0) = x_0 \in X$ and the homotopies preserve $x_0$; 
	\needfig{6}
	$\pi_1$ is called the \emph{fundamental group} on which the group operation is the concatenation $(\gamma, \tau) \mapsto \gamma * \tau$. 
	
	Similarly, for higher dimensions \needfig{7}

	giving
	\[
		\pi_n (X, x_0) = \{\text{based maps}\} / \simeq
	\]
	called the \emph{$n$-th homotopy group} of $X$.

	The issue is that these homotopy groups are very hard to compute. E.g.\ $\pi_n (S^2, x_0)$ is not known $\forall n$.
	
	There is even \emph{no} simply connected manifold (a space $X$ locally homeomorphic to $\mathbb{R}^n$) of dimension $> 0$ with $\pi_n (X)$ known $\forall n$!
	
	Therefore, we will do something else: \emph{(co)homology}.

	It is algebraically harder to set up, yet the computational gain is worth it. Please note that computing cohomology of ``harder'' spaces (e.g.\ $\text{Diff}(X), \text{Emb}(X,Y), \dots$) is still very hard.

	Some general remarks:
	\begin{itemize}
		\item Algebraic topology is all about being able to \emph{compute}. It is important to do lots of examples;
		\item Our ``nice spaces'' are \emph{manifolds} and indeed \emph{smooth manifolds} --- some of these will overlap with the course \emph{Differential Geometry} which will be useful. 
	\end{itemize}

	\newpage
	\section{Homology}
	\subsection{Chain \& Cochain Complexes} \lecnr{2}
\end{document}