\documentclass[a4paper,11pt]{article}

\usepackage{../../zyancamlec}

\def\ntripos{Mathematical Tripos}
\def\npart{III}
\def\ncourse{Statistical Field Theory}
\def\nscourse{SFT}
\def\nlecturer{C.\ Thomas}
\def\nterm{Michaelmas}
\def\nyear{2020}

\begin{document}
	\maketitlepage
	\preliminaries

	\section*{Introduction}
	
	This course introduces the renormalization group, focusing on statistical systems such as spin models with further connections to quantum field theory.

	\noindent After introducing the Ising Model, Landau’s mean field theory is introduced and used to describe phase transitions. The extension to the Landau-Ginzburg theory reveals broader aspects of fluctuations whilst consolidating connections to quantum field theory. At second order phase transitions, also known as ‘critical points’, renormalisation group methods play a starring role. Ideas such as scaling, critical exponents and anomalous dimensions are developed and applied to a number of different systems.

	\section*{Pre-requisites}

	Background knowledge of Statistical Mechanics at an undergraduate level is essential. This course complements the Quantum Field Theory and Advanced Quantum Field Theory courses.
  
	\newpage
	\tableofcontents
	\newpage
	\maintext
	Hi

	
\end{document}