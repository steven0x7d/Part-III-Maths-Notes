\documentclass[a4paper,11pt]{article}

\usepackage{../../zyancamlec}

\def\ntripos{Mathematical Tripos}
\def\npart{III}
\def\ncourse{Statistical Field Theory}
\def\nscourse{SFT}
\def\nlecturer{C.\ Thomas}
\def\nterm{Michaelmas}
\def\nyear{2020}

\begin{document}
	\maketitlepage
	\preliminaries

	\section*{Introduction}
	
	This course introduces the renormalization group, focusing on statistical systems such as spin models with further connections to quantum field theory.

	\noindent After introducing the Ising Model, Landau’s mean field theory is introduced and used to describe phase transitions. The extension to the Landau-Ginzburg theory reveals broader aspects of fluctuations whilst consolidating connections to quantum field theory. At second order phase transitions, also known as ‘critical points’, renormalisation group methods play a starring role. Ideas such as scaling, critical exponents and anomalous dimensions are developed and applied to a number of different systems.

	\section*{Pre-requisites}

	Background knowledge of Statistical Mechanics at an undergraduate level is essential. This course complements the Quantum Field Theory and Advanced Quantum Field Theory courses.

	\newpage

	\section*{Overview}
	\begin{enumerate}
		\item Phase Transitions (discontinuous, sudden change of properties)
		\item Critical Points, e.g.\ \[
			C \sim \frac{1}{|T - T_C|^{0.11008\dots}}
		\]
		\item Universality:
		\begin{itemize}
			\item Symmetry
			\item Physics of different length scales, Renormalisation Group
		\end{itemize}
	\end{enumerate}


  
	\newpage
	\tableofcontents
	\newpage
	\maintext
	\section{From Spin to Fields}
	\recnr{1-1}
	\subsection{The Ising Model}
	Think about a simple model for a magnet. Consider a lattice in $d$ spatial dimensions with $N$ sites. On each site $i = 1, \dots , N$ there is a discrete variable, called `spin', which takes values:
	\begin{align*}
		S_i = + 1 = \uparrow\\
		S_i = - 1 = \downarrow
	\end{align*}

	\needfig{Ising Lattice}

	The system of spins $\{S_i\}$ has energy
	\[
		E = -B \sum_i S_i - J \sum_{\expval{ij}}S_i S_j
	\]
	where $\expval{ij} = $ nearest neighbour pairs, $B$ is an external magnetic field:
	\begin{align*}
		B>0 \qquad \Rightarrow\qquad \text{spins prefer }\uparrow\\
		B<0 \qquad \Rightarrow\qquad \text{spins prefer }\downarrow
	\end{align*}
	\begin{align*}
		J > 0 &\qquad \Rightarrow\qquad \text{spins want to align}\\
		J < 0 &\qquad \Rightarrow\qquad \text{spins want to anti-align}
	\end{align*}
	and we call $\uparrow\uparrow$ ferromagnet and $\downarrow\uparrow$ anti-ferromagnet.
	
	In the scope of this course, we take $J > 0$.

	For a finite temperature $T$, spins want to randomise to increase entropy.

	For canonical ensemble, the probability of configuration $\{S_i\}$ is 
	$$P[S_i] = \frac{e^{- \beta E[S_i]}}{Z}$$ 
	where $Z(T,J,B) = \sum_{\{S_i\}} e ^{-\beta E[S_i]}$ is the \emph{partition function}, $\beta = 1/T$ and we choose units such that $k_B = 1$.

	All physics can be extracted from $Z$.

	\begin{ex} \ 
		\begin{itemize}
			\item Thermodynamic free energy $F_{\text{thermo}}(T,J,B) = \expval{E} - TS = - T \ln Z$. For this we have 
			\[
				\dd{F} = -S \dd{T} - p \dd{V} - M \dd{B}
			\]
			\[
				-S = \pdv{F}{T}\Bigg|_{V,B} \qquad -p = \pdv{F}{V}\Bigg|_{T,B} \qquad - M = \pdv{F}{B}\Bigg|_{T,V}
			\]
			\item Equilibrium magnetisation \[
				[-1,+1] \ni m = \frac{1}{N}\sum_i \expval{S_i} = \frac{1}{N}\sum_{\{S_i\}} \frac{e^{-\beta E[S_i]}}{Z} \sum_i S_i = \frac{1}{N\beta} \pdv{\ln Z}{\beta}
			\]
			We want to understand how this changes as we vary $T$ and $B$ (equation of state).
		\end{itemize}
	\end{ex}

	All we need to do is compute $Z$. But this is very difficult and sometimes impossible. We need a different approach... 

	\paragraph{\underline{Effective Free Energy}} \ 

	We rewrite $Z$ as
	\[
		Z = \sum_m \sum_{\{S_i\}|m} e^{-\beta E[S_i]} : = \sum_m e^{-\beta F(m)}
	\]
	where the expression $\{S_i\} | m$ means we choose $\{S_i\}$ such that $m = \frac{1}{N} \sum_i S_i$.

	For large $N ( N \approx 10^{23})$, $m$ is essentially continuous, we can write
	\[
		Z = \frac{N}{2} \int_{-1}^{+1} \dd{m} e^{-\beta F(m)}
	\]
	where $N/2$ is there to make the normalisation correct.

	It is easy to see that effective free energy $F(m,T,J,B)$ contains more information than $F_{\text{thermo}}$.

	We can determine the equilibrium value of $m$ from it. Define 
	\[
		f(m):= \frac{F(m)}{N} \qquad \text{such that} \qquad Z = \frac{N}{2} \int_{-1}^{+1} \dd{m} e^{\beta N f(m)}
	\]
	where $N$ is large and $\beta f(m)$ is of order 1.

	The above integral is dominated by the minimum value of $f(m)$ where $m = m_{\text{min}}$ at $\pdv{f}{m}\bigg|_{m = m_{\text{min}}} = 0$. This is known as the saddle point/steepest descent technique. With this approximation, we have
	\[
		Z \approx e^{-\beta N f(m_{\text{min}})} \qquad \text{so} \qquad F_{\text{thermo}} \approx F(m_{\text{min}})
	\]
	
	We want to compute $f(m)$ but this is also hard.

	\paragraph{\underline{Mean Field `Approximation'}} \ 

	For each $\{S_i\}$ s.t.\ $\sum_i S_i = Nm$, we guess that the energy is approximately
	\begin{align*}
		E & \approx - B \sum_i m - J \sum_{\expval{ij}} m^2\\
		& = - B N m - \frac{1}{2} N J q m^2
	\end{align*}
	where the factor of half comes from the fact that we are summing over pairs but not sites, and $q$ is the number of nearest neighbours. (E.g.\ $q=2$ for $d=1$, $q=4$ for $d=2$ square lattice, $q=6$ for $d=3$ simple cubic lattice.)

	Now we just need to count the number of configurations. If there are $N_{\uparrow}$ up spins and $N_{\downarrow} = N - N_{\uparrow}$ down spins, we have
	\[
		m = \frac{N_{\uparrow} - N_\downarrow}{N} = \frac{2 N_\uparrow - N}{N}
	\]
	and the number of configurations is
	\[
		\Omega = \frac{N!}{N_\uparrow ! (N - N_\downarrow)!}
	\]
	
	Using Stirling's formula
	\[
		\ln \Omega \approx N \ln N - N_\uparrow \ln N_\uparrow - (N - N_\uparrow) \ln (N - N_\uparrow) - N + N_\uparrow + N - N_\uparrow + \cdots
	\]
	so that
	\[
		\frac{1}{N} \ln \Omega = \ln 2 - \frac{1}{2} (1+m)\ln(1+m) - \frac{1}{2} (1-m)\ln(1-m)
	\]
	
	Therefore, in the mean field approximation, we have
	\begin{align*}
		e_{-\beta N f(m)} & = \sum_{\{S_i\}|m} e^{-\beta E[S_i]}\\
		& \approx \Omega(m) e^{-\beta E(m)}\\
		- \beta N f(m) & \approx \ln \Omega(m) - \beta E(m)
	\end{align*}

	Then we have
	\[
		f(m) \approx - Bm - \frac{1}{2} J q m^2 - \frac{1}{\beta} \left[ \ln 2 - \frac{1}{2} (1+m)\ln(1+m) - \frac{1}{2} (1-m)\ln(1-m) \right]
	\]
	
	Then to get the minimum value of $m$,
	\[
		\pdv{f}{m} = 0 \qquad \Rightarrow\qquad \beta (B + Jqm) = \frac{1}{2} \ln \left( \frac{1+m}{1-m} \right)
	\]
	and rearrange, we get
	\[
		m = \tanh [\beta (B + Jqm)]
	\]
	
	We can think about this as if
	\[
		B \to B_{\text{eff}} = B + J q m
	\]
	as the background magnetic field seen from some site can be seen as modified by the magnetisation of neighbouring sites.
	
\end{document}