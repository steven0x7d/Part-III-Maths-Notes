\documentclass[a4paper,11pt]{article}

\usepackage{../../zyancamlec}

\def\ntripos{Mathematical Tripos}
\def\npart{III}
\def\ncourse{Statistical Field Theory}
\def\nscourse{SFT}
\def\nlecturer{C.\ Thomas}
\def\nterm{Michaelmas}
\def\nyear{2020}

\begin{document}
	\maketitlepage
	\preliminaries

	\section*{Course Information}
	
	This course introduces the renormalization group, focusing on statistical systems such as spin models with further connections to quantum field theory.

	\noindent After introducing the Ising Model, Landau’s mean field theory is introduced and used to describe phase transitions. The extension to the Landau-Ginzburg theory reveals broader aspects of fluctuations whilst consolidating connections to quantum field theory. At second order phase transitions, also known as ‘critical points’, renormalisation group methods play a starring role. Ideas such as scaling, critical exponents and anomalous dimensions are developed and applied to a number of different systems.

	\section*{Pre-requisites}

	Background knowledge of Statistical Mechanics at an undergraduate level is essential. This course complements the Quantum Field Theory and Advanced Quantum Field Theory courses.

	\newpage

	\section*{Overview}
	\begin{enumerate}
		\item Phase Transitions (discontinuous, sudden change of properties)
		\item Critical Points, e.g.\ \[
			C \sim \frac{1}{|T - T_C|^{0.11008\dots}}
		\]
		\item Universality:
		\begin{itemize}
			\item Symmetry
			\item Physics of different length scales, Renormalisation Group
		\end{itemize}
	\end{enumerate}


  
	\newpage
	\tableofcontents
	\newpage
	\maintext
	\section{From Spin to Fields}
	\recnr{1-1}
	\subsection{The Ising Model}
	Think about a simple model for a magnet. Consider a lattice in $d$ spatial dimensions with $N$ sites. On each site $i = 1, \dots , N$ there is a discrete variable, called `spin', which takes values:
	\begin{align*}
		S_i = + 1 = \uparrow\\
		S_i = - 1 = \downarrow
	\end{align*}

	\needfig{Ising Lattice}

	The system of spins $\{S_i\}$ has energy
	\[
		E = -B \sum_i S_i - J \sum_{\expval{ij}}S_i S_j
	\]
	where $\expval{ij} = $ nearest neighbour pairs, $B$ is an external magnetic field:
	\begin{align*}
		B>0 \qquad \Rightarrow\qquad \text{spins prefer }\uparrow\\
		B<0 \qquad \Rightarrow\qquad \text{spins prefer }\downarrow
	\end{align*}
	\begin{align*}
		J > 0 &\qquad \Rightarrow\qquad \text{spins want to align}\\
		J < 0 &\qquad \Rightarrow\qquad \text{spins want to anti-align}
	\end{align*}
	and we call $\uparrow\uparrow$ ferromagnet and $\downarrow\uparrow$ anti-ferromagnet.
	
	In the scope of this course, we take $J > 0$.

	For a finite temperature $T$, spins want to randomise to increase entropy.

	For canonical ensemble, the probability of configuration $\{S_i\}$ is 
	$$P[S_i] = \frac{e^{- \beta E[S_i]}}{Z}$$ 
	where $Z(T,J,B) = \sum_{\{S_i\}} e ^{-\beta E[S_i]}$ is the \emph{partition function}, $\beta = 1/T$ and we choose units such that $k_B = 1$.

	All physics can be extracted from $Z$.

	\begin{ex} \ 
		\begin{itemize}
			\item Thermodynamic free energy $F_{\text{thermo}}(T,J,B) = \expval{E} - TS = - T \ln Z$. For this we have 
			\[
				\dd{F} = -S \dd{T} - p \dd{V} - M \dd{B}
			\]
			\[
				-S = \pdv{F}{T}\Bigg|_{V,B} \qquad -p = \pdv{F}{V}\Bigg|_{T,B} \qquad - M = \pdv{F}{B}\Bigg|_{T,V}
			\]
			\item Equilibrium magnetisation \[
				[-1,+1] \ni m = \frac{1}{N}\sum_i \expval{S_i} = \frac{1}{N}\sum_{\{S_i\}} \frac{e^{-\beta E[S_i]}}{Z} \sum_i S_i = \frac{1}{N\beta} \pdv{\ln Z}{B}
			\]
			We want to understand how this changes as we vary $T$ and $B$ (equation of state).
		\end{itemize}
	\end{ex}

	All we need to do is compute $Z$. But this is very difficult and sometimes impossible. We need a different approach... 

	\paragraph{\underline{Effective Free Energy}} \ 

	We rewrite $Z$ as
	\[
		Z = \sum_m \sum_{\{S_i\}|m} e^{-\beta E[S_i]} : = \sum_m e^{-\beta F(m)}
	\]
	where the expression $\{S_i\} | m$ means we choose $\{S_i\}$ such that $m = \frac{1}{N} \sum_i S_i$.

	For large $N ( N \approx 10^{23})$, $m$ is essentially continuous, we can write
	\[
		Z = \frac{N}{2} \int_{-1}^{+1} \dd{m} e^{-\beta F(m)}
	\]
	where $N/2$ is there to make the normalisation correct.

	It is easy to see that effective free energy $F(m,T,J,B)$ contains more information than $F_{\text{thermo}}$.

	We can determine the equilibrium value of $m$ from it. Define 
	\[
		f(m):= \frac{F(m)}{N} \qquad \text{such that} \qquad Z = \frac{N}{2} \int_{-1}^{+1} \dd{m} e^{\beta N f(m)}
	\]
	where $N$ is large and $\beta f(m)$ is of order 1.

	The above integral is dominated by the minimum value of $f(m)$ where $m = m_{\text{min}}$ at $\pdv{f}{m}\bigg|_{m = m_{\text{min}}} = 0$. This is known as the saddle point/steepest descent technique. With this approximation, we have
	\[
		Z \approx e^{-\beta N f(m_{\text{min}})} \qquad \text{so} \qquad F_{\text{thermo}} \approx F(m_{\text{min}})
	\]
	
	We want to compute $f(m)$ but this is also hard.

	\paragraph{\underline{Mean Field `Approximation'}} \ 

	For each $\{S_i\}$ s.t.\ $\sum_i S_i = Nm$, we guess that the energy is approximately
	\begin{align*}
		E & \approx - B \sum_i m - J \sum_{\expval{ij}} m^2\\
		& = - B N m - \frac{1}{2} N J q m^2
	\end{align*}
	where the factor of half comes from the fact that we are summing over pairs but not sites, and $q$ is the number of nearest neighbours. (E.g.\ $q=2$ for $d=1$, $q=4$ for $d=2$ square lattice, $q=6$ for $d=3$ simple cubic lattice.)

	Now we just need to count the number of configurations. If there are $N_{\uparrow}$ up spins and $N_{\downarrow} = N - N_{\uparrow}$ down spins, we have
	\[
		m = \frac{N_{\uparrow} - N_\downarrow}{N} = \frac{2 N_\uparrow - N}{N}
	\]
	and the number of configurations is
	\[
		\Omega = \frac{N!}{N_\uparrow ! (N - N_\downarrow)!}
	\]
	
	Using Stirling's formula
	\[
		\ln \Omega \approx N \ln N - N_\uparrow \ln N_\uparrow - (N - N_\uparrow) \ln (N - N_\uparrow) - N + N_\uparrow + N - N_\uparrow + \cdots
	\]
	so that
	\[
		\frac{1}{N} \ln \Omega = \ln 2 - \frac{1}{2} (1+m)\ln(1+m) - \frac{1}{2} (1-m)\ln(1-m)
	\]
	
	Therefore, in the mean field approximation, we have
	\begin{align*}
		e_{-\beta N f(m)} & = \sum_{\{S_i\}|m} e^{-\beta E[S_i]}\\
		& \approx \Omega(m) e^{-\beta E(m)}\\
		- \beta N f(m) & \approx \ln \Omega(m) - \beta E(m)
	\end{align*}

	Then we have
	\[
		f(m) \approx - Bm - \frac{1}{2} J q m^2 - \frac{1}{\beta} \left[ \ln 2 - \frac{1}{2} (1+m)\ln(1+m) - \frac{1}{2} (1-m)\ln(1-m) \right]
	\]
	
	Then to get the minimum value of $m$,
	\[
		\pdv{f}{m} = 0 \qquad \Rightarrow\qquad \beta (B + Jqm) = \frac{1}{2} \ln \left( \frac{1+m}{1-m} \right)
	\]
	and rearrange, we get
	\[
		m = \tanh [\beta (B + Jqm)]
	\]
	
	We can think about this as if
	\[
		B \to B_{\text{eff}} = B + J q m
	\]
	as the background magnetic field seen from some site can be seen as modified by the magnetisation of neighbouring sites.
	
	\subsection{Landau Theory of Phase Transitions} \recnr{1-2-1}

	A phase transition occurs when some quantity changes discontinuously. For us this \emph{order parameter} is $m$.

	Landau theory is a simple yet effective way to understand phase transitions qualitatively. Based on free energy and symmetry.

	Recall the free energy for the Ising model in the mean field approximation
	\[
		f(m) \approx - B m - \frac{1}{2} J q m^2 - \frac{1}{\beta} \left[ \ln 2 - \frac{1}{2}(1+m)\ln(1+m) - \frac{1}{2} (1-m)\ln(1-m) \right]
	\]
	
	Taylor expand for small $m$ and discarding the constant term (as it is not important), we get
	\[
		f(m)\approx -Bm + \frac{1}{2} \left( \frac{1}{\beta} - J q \right) m^2 + \frac{1}{12} \frac{1}{\beta}m^4 + \cdots
	\]
	
	The equilibrium value of $m$ is $m_{\text{min}}$. Landau theory investigates how $m_{\text{min}}$ changes as we vary the parameters.

	\subsubsection{$B=0$}
	\[
		f(m) \approx \frac{1}{2} (T - T_c) m^2 + \frac{1}{12} T m^4 + \cdots
	\]
	where we denoted the \emph{critical temperature} $T_c = Jq$.

	\needfig{2}

	At high $T$, we have $m=0$ as the minimum. For low $T$, specifically $T < T_c$, we have the minima
	\[
		m = \pm m_0 = \pm \sqrt{\frac{3(T_c - T)}{T}}
	\]
	and note that this holds only when close to $T_c$ where $m$ is small. 

	If we plot the phase diagram \needfig{3}, we find $m$ turns on abruptly as we lower $T$. Here, $m$ is continuous, so this is called a \emph{continuous phase transition} (also known as a $2^{\text{nd}}$ order phase transition).

	There are two phases:
	\begin{itemize}
		\item $m=0$ is the \emph{disordered phase};
		\item $m \neq 0$ is the \emph{ordered phase}.
	\end{itemize}

	$f(m)$ is invariant under a $\mathbb{Z}_2$ symmetry $m \to -m$, which is inherited from the $S_i \to -S_i$ symmetry of the Ising model when $B = 0$.

	For $T < T_c$, the system chooses one of the two states $m = \pm m_0$, i.e.\ the $\mathbb{Z}_2$ symmetry is \emph{spontaneously broken}.

	Consider the heat capacity at constant $B = 0$ (this is a response function):
	\[
		C = \pdv{\expval{E}}{T} = \beta^2 \pdv[2]{\beta} \ln Z
	\]
	
	Some algebra gives
	\[
		f(m_{\text{min}}) = \begin{dcases}
			0 & T > T_c \\
			- \frac{3}{4} \frac{(T_c - T)^2}{T} & T<T_c
		\end{dcases}
	\]
	
	Using $\ln Z \approx - \beta N f(m_{\text{min}})$ gives
	\[
		c = \frac{C}{N} \to \begin{dcases}
			0 & T \to T_c^+ \\
			3/2 & T \to T_c^-
		\end{dcases}
	\]
	
	This means $c$ jumps discontinuously.
	
	\needfig{4}

	\subsubsection{$B\neq 0$}
	\[
		f(m) \approx - B m + \frac{1}{2}(T - T_c)m^2 + \frac{1}{12} T m^4 + \cdots
	\]
	
	For high $T$, we have a unique minimum. For low $T$, we have two minima: one meta-stable, one true minimum.

	\needfig{5}

	As we vary $T$, the magnetisation varies smoothly
	\[
		m \to \frac{B}{T} \quad \text{as} \quad T \to \infty
	\]
	and
	\[
		m \to \pm 1 \quad \text{as} \quad T \to 0
	\]
	
	But now the sign of the magnetic field determines the sign of $m$. Thus no phase transition as we vary $T$.

	\needfig{6}

	If instead we fix $T$ (low $T$) and vary $B$, as $B$ passes through $0$, $m$ jumps discontinuously from $m = - m_0$ to $m = + m_0$ --- this is a \emph{first-order phase transition}.

	\needfig{7}

	The phase diagram is shown below.

	\needfig{8}

	What happens close to the critical point? If we are right at the critical point, 
	\[
		f(m) \approx - B m + \frac{1}{12}T_c m^4 + \cdots
	\]
	which gives
	\[
		m \sim B^{1/3} \quad (B > 0)
	\]
	
	Consider magnetic susceptibility $\chi = \pdv{m}{B}\big|_T$, which is another response function. 
	
	Then for $T>T_c$, 
	\[
		f(m) \approx -B m + \frac{1}{2} (T- T_c)m^2 + \cdots \quad \text{and} \quad m \approx \frac{B}{T - T_c}
	\]
	so
	\[
		\chi \sim \frac{1}{T - T_c} \quad \text{as} \quad T \to T_c^+
	\]
	
	For $T<T_c$, similarly,
	\[
		\chi \sim \frac{1}{\abs{T - T_c}} \quad \text{as} \quad T \to T_c^-
	\]

	\subsubsection{Validity of Mean Field Theory} \recnr{1-2-2}

	Are these results correct?
	\begin{itemize}
		\item $d = 1$: No. MFT is completely wrong! No phase transitions;
		\item $d = 2,3$: Basic structure of phase diagram is right, details near $T_c$ are wrong;
		\item $d \geq 4$: MFT gives right answers.
	\end{itemize}

	Similar story for other systems:
	\begin{itemize}
		\item MFT fails completely for $d \leq d_l$, the \emph{lower critical dimension};
		\item MFT always works for $d \geq d_u$, the \emph{upper critical dimension};
		\item What about $d_l < d < d_u$? Often these $d$ are interesting.
	\end{itemize}
	
	In Ising model, $d_l = 1$ and $d_u = 4$.

	\subsubsection{Critical Exponents}

	We've seen that various quantities scale as power laws as we approach the critical point.
	\begin{align*}
		m \sim (T_c - T)^\beta \qquad & \beta = 1/2 \quad (T < T_c)\\
		c \sim c_{\pm} \abs{T - T_c}^{-\alpha} \qquad & \alpha = 0 \text{ (discontinuity)}\\
		\chi \sim \abs{T - T_c}^{-\gamma} \qquad & \gamma = 1\\
		m \sim B^{1/\delta} \qquad & \delta = 3
	\end{align*}

	Compare MFT results with true values for Ising model in $d = 2,3$, we have
	
	\begin{center}
		\begin{tabular}{c|ccc}
			& MFT & $d=2$ & $d=3$ \\
			\hline
			$\alpha$ & 0 (disc.) & 0 (log div.) & 0.1101...\\
			$\beta$ & 1/2 & 1/8 & 0.3264...\\
			$\gamma$ & 1 & 7/4 & 1.2371...\\
			$\delta$ & 3 & 15 & 4.7898...
		\end{tabular}
	\end{center}

	\subsubsection{Universality}

	Liquid-gas transition shows many similarities to the Ising model.
	\needfig{9}

	Again, a line of first-order phase transitions ending at a critical point.

	At fixed $p$, $v = V/N$ jumps discontinuously at liquid-gas first-order phase transition. Suggests $v$ is order parameter analogous to $m$ in Ising model.

	Using an equation of state (e.g.\ van der Waals equation), at the critical point:
	\begin{align*}
		v_{\text{gas}} - v_{\text{liquid}} \sim (T_c - T)^{\beta} \qquad & \beta = 1/2 \quad (T < T_c)\\
		v_{\text{gas}} - v_{\text{liquid}} \sim (p - p_c)^{1/\delta} \qquad & \delta = 3 \quad(p \to p_c^+)\\
		\text{Compressibility, } \kappa = - \frac{1}{v}\pdv{v}{p}\bigg|_{T}\sim \abs{T - T_c}^{-\gamma} \qquad & \gamma = 1\\
		\text{Heat capacity, } C_V \sim C_{\pm}\abs{T - T_c}^{-\alpha} \qquad & \alpha = 0
	\end{align*}
	
	None of these agree with experiment.

	All liquid-gas transition show the same critical exponents, and these agree with $d = 3$ Ising.

	\emph{Universality}: critical point governs behaviour of many different physical systems. Why?!

	\subsubsection{Ising Model as Lattice Gas}

	Consider the same $d$-dimensional lattice, but now with particles hopping between sites.

	The number of particles on site is $n_i$, and we allow no more than one particle on each site, i.e.\ either $n_i = 0$ (empty) or $n_i = 1$ (filled). Then we have
	\[
		E = - 4 J \sum_{\expval{ij}}n_i n_j - \mu \sum_i n_i
	\]
	where $J > 0$ is acting as an attractive force and $\mu$ is the chemical potential.

	This is equivalent to the Ising model if we identify $S_i = 2n_i - 1$. 

	\subsection{Landau-Ginzburg Theory} \recnr{1-3}

	The Landau-Ginzberg (LG) approach is an extension of Landau's idea that, treated correctly, gives the correct physics.

	We now allow $m$ to vary in space: $m \to m(\vb x)$. It is now a field.

	Start with a lattice with $N$ sites, and divide into boxes each with $N'$ sites such that $1 \ll N' \ll N$.
	
	\needfig{10}

	For each box
	\[
		m(\vb x) = \frac{1}{N'} \sum_i S_i
	\]
	where $\vb x$ is the centre of the box and the sum is only over the sites in box.

	This is a \emph{coarse graining}. We require
	\begin{itemize}
		\item $N'$ is large enough so $m \in [-1,+1]$ is essentially continuous;
		\item Box side length $a \ll$ other relevant scales so $\vb x$ can be treated as continuous. But we must remember that $m$ cannot vary on scales $<a$.
	\end{itemize}

	Write the partition function as 
	\[
		Z = \sum _{m(\vb x)} \sum _{\{S_i\}|m(\vb x)} e ^{- \beta E[S_i]} := \sum _{m(\vb x)} e ^{-\beta F[m(\vb x)]}
	\]
	where the summation is over all $\{S_i\}$ such that coarse graining gives $m(\vb x)$ and we defined the free energy $F[m(\vb x)]$ to be a functional (that takes a function $m(\vb x)$ and gives a number).

	Now sum is over all possible fields $m(\vb x)$.

	We write $Z$ as a \emph{functional integral}, also known as a \emph{path integral}.
	\[
		Z = \int \ddp m(\vb x) e ^{-\beta F[m(\vb x)]}
	\]
	
	The probability of field configuration $m(\vb x)$ is
	\[
		P[m(\vb x)] = \frac{e ^{- \beta F[m(\vb x)]}}{Z}.
	\]
	
	\subsubsection{Free Energy Functional}

	What is the free energy functional $F[m(\vb x)]$?

	It must have the following properties:
	\begin{itemize}
		\item \textbf{Locality}: in the Ising model spins only affect their neighbours \[
			F[m(\vb x)] = \int \dd[d]{\vb x} f[m(\vb x)]
		\]
		where $f[m(\vb x)]$ depends only on $m(\vb x), \grad m(\vb x), \laplacian m(\vb x), \cdots$;
		\item \textbf{Translation and rotation symmetry}: inherited from discrete versions of these symmetries on the lattice;
		\item A $\mathbb{Z}_2$ \textbf{symmetry} when $B = 0$: $m(\vb x) \to - m(\vb x)$ and when $B \neq 0$ there's a symmetry $B \to - B, m(\vb x) \to - m(\vb x)$;
		\item \textbf{Analyticity}: we are interested in the critical point when $m$ is small. Assume $f[m(\vb x)]$ has a Taylor series expansion;
		\item We are interested in $m(\vb x)$ that are slowly varying in space, e.g.\ $\nabla m$ terms are more important than $a \laplacian m$ terms, etc.
	\end{itemize}

	Most general free energy when $B = 0$ writes
	\[
		F[m(\vb x)] = \int \dd[d]{\vb x} \left[ \frac{1}{2} \alpha_2(T) m^2 + \frac{1}{4} \alpha_4(T) m^4 + \frac{1}{2} \gamma(T) (\nabla m)^2 + \cdots \right].
	\]
	
	Note that we only have even powers in $m$ as odd powers break the $\mathbb{Z}_2$ symmetry. Similar for $\nabla m$ terms as odd ones kills the rotational symmetry off.

	When $B \neq 0$ we can have $Bm, Bm^3$, etc.\ terms.

	It's hard to compute coefficients $\alpha_2(T), \alpha_4(T), \gamma(T)$ from first principles. From MFT we expect
	\begin{align*}
		\alpha_2(T) & \sim (T - T_c),\\
		\alpha_4(T) & \sim \frac{1}{3}T.
	\end{align*}
	But we only need
	\[
		\alpha_4(T), \gamma(T) > 0 \quad \text{and} \quad \alpha_2(T) \text{ changes sign at } T = T_c.
	\]
	
	\subsubsection{Saddle Point}
	For the path integral 
	\[
		Z = \int \ddp m(\vb x) e^{-\beta F[m(\vb x)]},
	\]
	our first guess is to assume integral is dominated by the saddle point.

	Consider $m(\vb x) + \delta m(\vb x)$, we get
	\begin{align*}
		\delta F & = \int \dd[d]{\vb x} [\alpha_2 m\ \delta m + \alpha_4 m^3\ \delta m + \gamma (\grad m) \vdot (\grad \delta m) + \mathcal{O}((\delta m)^2)]\\
		& = \int \dd[d]{\vb x} [\alpha_2 m + \alpha_4 m^3 - \gamma \nabla^2 m] (\delta m) + \cdots \quad \text{[integrated by parts]}
	\end{align*} 

	This has minimum when
	\[
		\gamma \laplacian m = \alpha_2 m + \alpha_4 m^3.
	\]
	
	The simplest solution is $m(\vb x) = m = \text{constant}$ which reproduces results of Landau theory
	\begin{align*}
		& T > T_c \Rightarrow \alpha_2 > 0 \Rightarrow m = 0\\
		& T < T_v \Rightarrow \alpha_2 < 0 \Rightarrow m = \pm m_0 = \pm \sqrt{\frac{-\alpha_2}{\alpha_4}} = \pm \sqrt{\frac{3(T_c - T)}{T}}.
	\end{align*}

	\subsubsection{Domain Walls}

	For $T < T_c$, there are two ground states $m \pm m_0$.

	Consider state where part of space has $m = + m_0$ and the other part has $m = - m_0$.

	Two different `\emph{domains}' separated by a `\emph{domain wall}'.

	\needfig{11}

	Write $m(\vb x) = m(x)$ since we here only vary the $x$-coordinate. We get
	\[
		\gamma \dv[2]{m}{x} = \alpha_2 m + \alpha_4 m^3
	\]
	with boundary conditions $m(x)\to \pm m_0$ as $x \to \pm \infty$.
	
	This has solution
	\[
		m = m_0 \tanh(\frac{x - x_0}{W})
	\]
	where $x_0$ is the position of the domain wall and $W = \sqrt{-\gamma/\alpha_2}$ is the width of the wall.
	
	\needfig{12}

	If linear size of system is $L$, the excess cost in free energy is
	\[
		F_{\text{wall}} \sim L ^{d-1} \int \dd{x} \gamma \left( \dv{m}{x} \right)^2 \sim L ^{d-1} W \gamma \left( \frac{m_0}{W} \right)^2.
	\]
	where we regard $L ^{d-1}$ as the `area' of the domain wall. Plug $W$ in, we have
	\[
		F _{\text{wall}} \sim L ^{d-1} \sqrt{\frac{- \gamma \alpha_2^3}{\alpha_4^2}}.
	\]
	
	\subsubsection{Lower Critical Dimension}

	It turns out that domain walls are responsible for the lack of phase transition for $d = 1$.

	Suppose the system is in the ordered phase, $\alpha_2(T)<0$, and consider an interval of length $L$.

	Suppose at $x = -L/2$ we fix $m = + m_0$, we want to calculate the probability of $m = + m_0$ at the other end $x = + L/2$.

	\needfig{13}

	A domain wall flips the state, and
	\[
		P[\text{wall at } x = x_0] = \frac{e ^{-\beta F _{\text{wall}}}}{Z}
	\]
	and
	\[
		P[\text{wall anywhere}] \sim \frac{L}{W} \frac{e ^{-\beta F _{\text{wall}}}}{Z}
	\]

	Suppose we have $n$ domain walls, if $n$ is even, it eventually gets back to $+m_0$; for $n$ is odd, it ends up at $- m_0$.
	
	\[
		P[n \text{ walls}] = \frac{e ^{-n \beta F _{\text{wall}}}}{Z} \frac{1}{W^n} \int_{-L/2}^{L/2} \dd{x_1} \int_{x_1}^{L/2} \dd{x_2} \cdots \int_{x _{n-1}}^{L/2} \dd{x_n} = \frac{1}{Z n!} \left( \frac{L e ^{- \beta F _{\text{wall}}}}{W} \right)^n
	\]
	
	Now we can answer the question
	\[
		P[m \to + m_0] = \frac{1}{Z} \sum _{n \text{ even}} \frac{1}{n!} \left( \frac{L e ^{- \beta F _{\text{wall}}}}{W} \right)^n = \frac{1}{Z} \cosh(\frac{L e ^{- \beta F _{\text{wall}}}}{W})
	\]
	and
	\[
		P[m \to - m_0] = \frac{1}{Z} \sum _{n \text{ odd}} \frac{1}{n!} \left( \frac{L e ^{- \beta F _{\text{wall}}}}{W} \right)^n = \frac{1}{Z} \sinh(\frac{L e ^{- \beta F _{\text{wall}}}}{W})
	\]

	As $L \to \infty$, $P[m \to + m_0] = P[m \to - m_0]$, i.e.\ all memory of the starting point gets washed away.

	Fluctuations of domain walls are unsuppressed because of entropy --- they proliferate.

	Thus there is no ordered phase and no spontaneous symmetry breaking in $d = 1$.
	
	This suggests (although we haven't proved yet) the lower critical dimension for Ising model is $d_l = 1$.

	

\end{document} 