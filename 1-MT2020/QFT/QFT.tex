\documentclass[a4paper,11pt]{article}

\usepackage{../../zyancamlec}
\usepackage[compat=1.1.0]{tikz-feynhand}

\def\ntripos{Mathematical Tripos}
\def\npart{III}
\def\ncourse{Quantum Field Theory}
\def\nscourse{QFT}
\def\nlecturer{N.\ Dorey}
\def\nterm{Michaelmas}
\def\nyear{2020}


\begin{document}
	\maketitlepage
	\preliminaries

	\section*{Course Information}
	
	\noindent Quantum Field Theory is the marriage of quantum mechanics with special relativity and provides the mathematical framework in which to describe the interactions of elementary particles.

	\noindent This first Quantum Field Theory course introduces the basic types of fields which play an important role in high energy physics: scalar, spinor (Dirac), and vector (gauge) fields. The relativistic invariance and symmetry properties of these fields are discussed using the language of Lagrangians and Noether’s theorem.

	\noindent The quantisation of the basic non-interacting free fields is firstly developed using the Hamiltonian and canonical methods in terms of operators which create and annihilate particles and anti-particles. The associated Fock space of quantum physical states is explained together with ideas about how particles propagate in spacetime and their statistics.

	\noindent Interactions betweeen fields are examined next, using the interaction picture, Dyson’s formula and Wick’s theorem. A ‘short version’ of these techniques is introduced: Feynman diagrams. Decay rates and interaction cross-sections are introduced, along with the associated kinematics and Mandelstam variables.

	\noindent Spinors and the Dirac equation are explored in detail, along with parity and $\gamma^{5}$. Fermionic quantisation is developed, along with Feynman rules and Feynman propagators for fermions.

	\noindent Finally, quantum electrodynamics (QED) is developed. A connection between the field strength tensor and Maxwell’s equations is carefully made, before gauge symmetry is introduced. Lorentz gauge is used as an example, before quantisation of the electromagnetic field and the Gupta-Bleuler condition. The interactions between photons and charged matter is governed by the principal of minimal coupling. Finally, an example QED cross-section calculation is performed.

	\section*{Pre-requisites}

	You will need to be comfortable with the Lagrangian and Hamiltonian formulations of classical mechanics and with special relativity. You will also need to have taken an advanced course on quantum mechanics.

	\newpage

	\section*{Introduction}
	\subsection*{Why do we need QFT?}
	If we consider the description of the interaction between two charged particles, for an accurate prediction, we need 
	\begin{itemize}
		\item \emph{Maxwell's theory} using a field theoretic perspective (using electric field $\vb{E}(\vb x, t)$ and magnetic field $\vb B(\vb x, t)$), which embodies \emph{locality} and encodes \emph{Lorentz invariance};
		\item And \emph{quantum mechanics} as the charged particles are too small to be considered classically. This encodes the \emph{``discrete''} and \emph{probabilistic} nature.
	\end{itemize}

	The reconcile of special relativity and quantum mechanics leads to \emph{Quantum Field Theory}, where new phenomena emerge:
	\begin{itemize}
		\item Particle creation;
		\begin{center}
			\begin{tikzpicture}[baseline=-5pt]
				\begin{feynhand}
					\vertex (i) at (0,0) {$\gamma$};
					\vertex (f1) at (2.5,1) {$e^+$};
					\vertex (f2) at (2.5,-1) {$e^-$};
					\vertex [particle] (a) at (1.5,0);

					\propag [pho] (i) to [] (a);
					\propag [antfer] (a) to [] (f1);
					\propag [fer] (a) to [] (f2);
				\end{feynhand}
			\end{tikzpicture}
			$\quad$ with $\quad E_{\gamma} \geq 2 m_e c^2$
		\end{center}
		\item Bose/Fermi statistics;
		\item And more...
	\end{itemize}

	It is typical to consider experiments where some initial state $\ket{i}$ transitions to $\ket{f}$.
	
	\needfig{1}

	The challenge we are facing is to calculate the probability 
	\[
		P_{i \to f} = \abs{\mathcal{A}_{i \to f}}^2
	\]
	where $A_{i \to f} \in \mathbb{C}$ is the amplitude of such transition.

	The properties of such probability are
	\begin{itemize}
		\item Unitarity: $\sum_f P_{i \to f} = 1$;
		\item Lorentz covariance.
	\end{itemize}

	\subsection*{What is QFT?}
	Instead of QM, we start from classical field theory.

	For a classical field $\phi(\vb x, t)$, we can use \emph{canonical quantisation} analogous to that in particle QM to obtain a quantum field operator $\hat{\phi}(\vb x,t)$. We will mainly follow this route in this course.

	Quantum field operators are very complicated things --- they are operator-valued functions in space and time.

	There are some key facts in QFT:
	\begin{itemize}
		\item Eigenstates of \emph{Hamiltonian} $\hat{H}[\hat{\phi}, \hat{\pi}]$ describes \emph{multi-particle states}. (We can obtain the eigenstates at least in a free field theory);
		\item We can calculate \[
			\mathcal{A}_{i \to f} = \mel{f}{e^{\mathrm{i} \hat{H} T}}{i}.
		\]
	\end{itemize}

	The remarkable feature of field theories is that there are very few which they can be consistently calculated. Thus we are left to limited, confined choices, which is an attractive character QFTs. Among these, \emph{gauge theories} are a special type, using which we can successfully describe our universe to some extent.

	\subsection*{Outline}
	\begin{enumerate}
		\item Classical Field Theory
		\item Free QFT
		\item Interacting QFT
		\item Fermions
		\item QED
	\end{enumerate}

	\newpage
	
	\tableofcontents
	\newpage
	\maintext
	\setcounter{section}{-1}
	\section{Preliminaries}
	\lecnr{1}
	\subsection{Units}
	In SI units, we know the dimensions of several fundamental constants are
	\begin{align*}
		c & \sim LT^{-1}\\
		\hbar & \sim L^2 M T^{-1}\\
		G & \sim L^3 M^{-1} T^{-2}
	\end{align*}

	In this course, we choose \emph{natural units} such that
	\[
		\hbar = c = 1
	\]
	
	Note that we are eliminating several dimensions identified in SI units, thus all quantities scale with some power of \emph{mass} or \emph{energy}.
	\[
		X \sim M^\delta
	\]
	where we call $\delta$ the dimension of the quantity and write
	\[
		[X] = \delta
	\]
	
	\begin{ex}
		\[
			[E] = +1 \qquad [L] = -1
		\]
	\end{ex}

	\subsection{Relativity}

	Unless otherwise stated, we use Einstein summation convention throughout the course.

	We work in Minkowski spacetime $\mathbb{R}^{3,1}$ with metric tensor
	\[
		\eta_{\mu\nu} = \diag \{+1, -1, -1, -1\}
	\]
	with the above conventional signature.

	We denote the coordinates in spacetime as 
	\[
		x^\mu = (t, \vb x)
	\]
	and using the metric, we have
	\[
		x_\mu = \eta_{\mu\nu} x^\nu
	\]
	known as ``lowering'' the indices. 
	
	Similarly, for ``raising'' the indices we use the inverse metric tensor $\eta^{\mu\nu}$ defined by
	\[
		\eta^{\mu\nu} \eta_{\nu\rho} = \delta^\mu_\rho
	\]
	\newpage
	\section{Classical Field Theory}

	\subsection{Lorentz Covariant Fields}

	Firstly, let's recall that the Minkowski metric is invariant under Lorentz transformations. And we consider the Lorentz transformation of coordinates:
	\begin{equation}
		x^\mu \mapsto (x')^\mu = \LT{\mu}{\nu}x^\nu
	\end{equation}
	where $\LT{\mu}{\nu}$ is a $4\times 4$ matrix encoding the Lorentz transformation. To find the properties of such $\Lambda$, we write
	\begin{equation}
		\LT{\mu}{\sigma}\LT{\nu}{\tau} \eta^{\sigma\tau} = \eta^{\mu\nu}
	\end{equation}

	Also, we exclude the time-reversal transformations by imposing
	\[
		\det \Lambda = + 1
	\]
	
	The conditions above fix \emph{proper} Lorentz transformations. These have 6 degrees of freedom: 3 rotations + 3 boosts. 

	Lorentz transformations form a group under composition. It is actually a Lie group, called \emph{Lorentz group}, denoted as $\mathrm{SO}(3,1)$.
	
	Now we bring out the main characters --- \emph{fields}.

	\begin{defi}
		A \emph{scalar field} is a function $\phi(x) = \phi(t, \vb x)$
		\[
			\phi : \underbrace{\mathbb{R}^{3,1}}_{\text{spacetime}} \to \underbrace{ \mathbb{R}}_{\text{field space}}
		\]
		such that it transforms as 
		\begin{equation}
			\phi(x) \to \phi'(x) := \phi(\Lambda^{-1} \cdot x)
		\end{equation}
		under (active) Lorentz transformation.
	\end{defi}

	\begin{nt}
		By the group property of Lorentz transformations, we can write
		\[
			\left( \Lambda^{-1} \cdot x \right)^\mu = \left( \Lambda^{-1} \right)\indices{^\mu_\nu} x^\nu
		\]
		where
		\[
			\left( \Lambda^{-1} \right)\indices{^\mu_\nu} \LT{\nu}{\rho} = \delta_\rho^\mu
		\]

		Often we denote $\left( \Lambda^{-1} \right)\indices{^\mu_\nu}$ as $\iLT{\nu}{\mu}$.
	\end{nt}

	Now consider the spacetime derivatives of some scalar field $\phi$:
	\[
		\partial_\mu \phi(x) := \pdv{\phi(x)}{x^\mu}
	\]

	We find it transforms as 
	\[
		\partial_\mu \phi(x) \to \iLT{\mu}{\nu} \partial_\nu \phi(\Lambda^{-1} \cdot x)
	\]
	
	We can raise the index as 
	\[
		\partial^\mu \phi(x) = \eta^{\mu\nu} \partial_\nu \phi(x)
	\]
	
	\begin{exer}
		Show $\partial^\mu \phi(x)$ transforms as a \emph{4-vector field} such that
		\[
			\partial^\mu \phi(x) \to \LT{\mu}{\nu} \partial^\nu \phi(\Lambda^{-1} \cdot x)
		\]
	\end{exer}

	The consequence is that 
	\[
		\partial_\mu \phi(x) \partial^\mu \phi(x) = \eta^{\mu\nu} \partial_\mu \phi(x) \partial_\nu \phi(x)
	\]
	transforms as a scalar field. That is
	\[
		\partial_\mu \phi \partial^\mu \phi(x) \to \partial_\mu \phi\partial^\mu \phi(\Lambda^{-1} \cdot x)
	\]
	
	\subsection{Lagrangian Formulation} \lecnr{2}

	\subsubsection{Review of Lagrangian Formulation}

	Here we consider a non-relativistic particle with mass $m$ and moving in a potential $V(q)$. Here we use $q = q(t)$ to denote its position at time $t$. Then its Lagrangian is
	\begin{equation}
		L(t) = L\left( q(t), \dot{q}(t) \right) = \frac{1}{2} \dot{q}^2 - V(q)
	\end{equation}

	The \emph{action} in time interval $[t_i, t_f]$ of this Lagrangian $L$ is defined to be 
	\begin{equation}
		S[q] := \int_{t_i}^{t_f} \dd{t} L \left( q(t),\dot{q}(t) \right)
	\end{equation}
	
	The equation(s) of motion can be obtained by the \emph{principle of least action}. The plan is
	\begin{itemize}
		\item We vary the path of the particle as $q(t) \to q(t) + \delta q(t)$ and fix the end points by $\delta q(t_i) = \delta(q_f) = 0$.
		\item The equations of motion, known as \emph{Euler-Lagrange equations} are obtained by imposing the condition that the action is stationary, i.e. \[
			\delta S = 0.
		\]
	\end{itemize}

	The above principle can be easily and clearly generalised in later contexts.

	For this specific example, we get the Euler-Lagrange equation
	\begin{equation}
		\ddot{q} = - \pdv{V}{q}.
	\end{equation}
	
	\subsubsection{Scalar Field Theories}
	
	To construct such a scalar field theory of physical significance, we require
	\begin{itemize}
		\item It should have \emph{Lorentz invariant} action;
		\item It has \emph{locality}, i.e.\ no coupling between fields (and their spacetime derivatives) at different points;
		\item It has \emph{at most two time derivatives}.
	\end{itemize}

	From now on, for convenience, we use $x$ to denote some point $(\vb x,t)$ in spacetime.

	\begin{defi}
		The \emph{Lagrangian} of a scalar field $\phi(x)$ is some time-dependent functional of $\phi$ and its derivatives, in the form of
		\begin{equation}
			L(t) = L [\phi, \partial_\mu \phi] = \int \dd[3]{x} \mathcal{L}(\phi(x),\partial_\mu \phi(x))
		\end{equation}
		where $\mathcal{L}$ is called the \emph{Lagrangian density}, which itself is a function of $\phi$ and its derivatives.
	\end{defi}

	\begin{nt}
		The proper `Lagrangian' is in fact barely used in the context. Instead, the `Lagrangian density' appears far more often. Thus, we usually abuse the terminology and refer to `Lagrangian density' just as `Lagrangian'.
	\end{nt}

	\begin{defi}
		The \emph{action} of a scalar field $\phi(x)$ for some time interval $[t_i,t_f]$ is a functional of the field and its derivatives in the form of
		\begin{equation}
			S_{t_i,t_f}[\phi, \partial_\mu \phi] = \int_{t_i}^{t_f} \dd{t} L[\phi, \partial_\mu \phi] = \int_{t_i}^{t_f} \dd{t} \int \dd[3]{x} \mathcal{L}.
		\end{equation}

		Practically, we often choose infinite time interval $t_i \to - \infty, t_f \to +\infty$, thus
		\begin{equation}
			S[\phi, \partial_\mu \phi] = \int_{\mathbb{R}^{3,1}} \dd[4]{x} \mathcal{L}(\phi(x),\partial_\mu \phi(x)).
		\end{equation}
	\end{defi}

	Under Lorentz transformation 
	\[
		x^\mu \to x'^\mu = \LT{\mu}{\nu} x^\nu
	\]
	we require that
	\[
		\mathcal{L}(x) = \mathcal{L}(\phi(x), \partial_\mu \phi(x))
	\]
	transforms as a scalar field, i.e.\ 
	\[
		\mathcal{L}(x) \to \mathcal{L}(\Lambda^{-1} \cdot x)
	\]
	
	Thus, changing variables by $y^\mu = \iLT{\nu}{\mu}x^\nu$ and noting $\det \Lambda = +1$ we have
	\[
		S \to S' = \int \dd[4]{x} \mathcal{L}(\Lambda^{-1} \cdot x) = \int \dd[4]{y} \mathcal{L}(y) = S,
	\]
	i.e.\ the action is invariant under Lorentz transformation.

	Recall our requirements for any physical scalar field, the most general Lagrangian of some scalar field $\phi$ is of the form
	\begin{equation}
		\mathcal{L} = \frac{1}{2} F(\phi) \partial_\mu \phi \partial^\mu \phi - V(\phi)
	\end{equation}
	for some scalar function $V(\phi)$.

	\begin{nt}
		We don't need to consider terms like $\phi \partial_\mu \partial^\mu \phi$, as they are the same as $\partial_\mu \phi \partial^\mu \phi$ up to some surface term (which has no contribution to the action).
	\end{nt}

	Now we generalise the principle of least action to classical (scalar) field theories.

	\begin{pos}[Principle of least action]
		The equations of motion (i.e.\ Euler-Lagrange equations) of a scalar field $\phi(x)$ should make the action $S[\phi,\partial_\mu \phi]$ \emph{stationary}, i.e.\ $\delta S = 0$, under variations $\phi(x) \to \phi(x) + \delta \phi(x)$ subject to the boundary condition $\delta \phi(x) = \delta \phi(t,\vb x) \to 0$ as $\abs{\vb x} \to \infty$ or $t \to \pm \infty$.
	\end{pos}

	To get the equation(s) of motion of $\phi$, we vary the action and set it to zero, by
	\begin{align*}
		\delta S & = \int \dd[4]{x} \left[ \pdv{\mathcal{L}}{\phi} \delta \phi + \pdv{\mathcal{L}}{(\partial_\mu \phi)}\delta (\partial_\mu \phi) \right]\\
		& = \int \dd[4]{x} \left[ \pdv{\mathcal{L}}{\phi} - \partial_\mu \left( \pdv{\mathcal{L}}{(\partial_\mu \phi)} \right) \right] \delta \phi + \underbrace{\int \dd[4]{x} \partial_\mu \left( \pdv{\mathcal{L}}{(\partial_\mu \phi)} \delta \phi \right)}_{ = \int_{\partial \mathbb{R}^{3,1}} \dd{S_\mu} \pdv{\mathcal{L}}{(\partial_\mu \phi)} \delta \phi = 0}.
	\end{align*}

	Thus, by setting $\delta S = 0$, we have, for any variation $\delta \phi$, the Euler-Lagrange equation
	\begin{equation}
		\pdv{\mathcal{L}}{\phi} - \partial_\mu \left( \pdv{\mathcal{L}}{(\partial_\mu \phi)} \right) = 0.
	\end{equation}

	For our `ansatz' with $F \equiv 1$, 
	\begin{equation}
		\mathcal{L} = \frac{1}{2} \partial_\mu \phi \partial^\mu \phi - V(\phi)
	\end{equation}
	we have
	\[
		\pdv{\mathcal{L}}{\phi} = - V'(\phi) := \dv{V}{\phi} \quad \text{and} \quad \pdv{\mathcal{L}}{(\partial_\mu \phi)} = \partial^\mu \phi.
	\]
	so that the Euler-Lagrange equation is
	\begin{equation}
		\partial_\mu \partial^\mu \phi + V'(\phi) = 0.
		\label{eq:ELgen}
	\end{equation}
	
	In general it is a non-linear partial differential equation.

	Let's see a special case of particular importance.
	\begin{ex}
		Klein-Gordon field theory has potential term
		\[
			V(\phi) = \frac{1}{2} m^2 \phi^2
		\]
		and the equation of motion is called \emph{Klein-Gordon equation}
		\begin{equation}
			\boxed{\partial_\mu \partial^\mu \phi + m^2 \phi = 0}.
		\end{equation}

		If we write the derivatives in terms of space and time explicitly, we have
		\[
			\partial_\mu \partial^\mu = \pdv[2]{t} - \sum_i \pdv[2]{x_i} = \pdv[2]{t} - \laplacian
		\]
		and the Klein-Gordon equation is actually a wave equation. It is a linear partial differential equation and has wavelike solutions such as 
		\[
			\phi \sim e^{\mathrm{i} p \cdot x} = e^{\mathrm{i} \omega t - \mathrm{i} \vb k \vdot \vb x}
		\]
		with dispersion relation
		\[
			\omega_{\vb k} = \sqrt{\abs{\vb k}^2 + m^2}.
		\]
		
		These solutions can be superposed, giving certain wave packets. (Think about Fourier transform.)
	\end{ex}

	\lecnr{3}
	
	Recall our general Euler-Lagrange equation (\ref{eq:ELgen}), it is a non-linear PDE. There won't be a general superposition principle. An exotic example is \emph{soliton}.
	
	\subsection{Maxwell's Theory: An Example of Classical FT}
	In Maxwell's theory, the central idea we use is a 4-vector potential
	\begin{equation*}
		A^\mu (x) = A^\mu(t, \vb x) = (\phi , \vb A)
	\end{equation*}
	which transforms under Lorentz transformation as 
	\begin{equation}
		A^\mu(x) \to \LT{\mu}{\nu} A^\nu (\Lambda^{-1} \cdot x).
	\end{equation}

	The electromagnetic fields live in the field strength tensor
	\begin{equation}
		F^{\mu\nu}(x) := \partial^\mu A^\nu(x) - \partial^\nu A^\mu(x)
	\end{equation}

	Under the Lorentz transformation, $F^{\mu\nu}$ transforms as 
	\begin{equation}
		F^{\mu\nu}(x) \to \LT{\mu}{\alpha}\LT{\nu}{\beta}F^{\alpha\beta}(\Lambda^{-1} \cdot x).
	\end{equation}

	$F^{\mu\nu}$ has some properties:
	\begin{itemize}
		\item We will see that $F^{\mu\nu}$ is invariant under \emph{gauge transformations} in the form 
		\begin{equation}
			A^\mu(x) \to A^\mu(x) + \partial^\mu \lambda(x)
		\end{equation} 
		where $\lambda(x)$ is an arbitrary function $\lambda : \mathbb{R}^{3,1} \to \mathbb{R}$.
		\item $F_{\mu,\nu}$ satisfies \emph{Bianchi identity}  \begin{equation}
			\partial_\lambda F_{\mu\nu} + \partial_\mu F_{\nu\lambda} + \partial_\nu F_{\lambda\mu} = 0.
		\end{equation}
	\end{itemize}

	The \emph{Maxwell Lagrangian} is 
	\begin{equation}
		\mathcal{L} = - \frac{1}{4} F_{\mu\nu} F^{\mu\nu} = - \frac{1}{2} (\partial_\mu A_\nu)(\partial^\mu A^\nu) + \frac{1}{2} (\partial_\mu A^\mu)^2 + \text{total derivatives}.
	\end{equation}

	Then the Euler-Lagrange equations are
	\[
		\pdv{\mathcal{L}}{A_\nu} - \partial_\mu \left( \pdv{\mathcal{L}}{(\partial_\mu A_\nu)} \right) = 0
	\]
	and by noting
	\[
		\pdv{\mathcal{L}}{A_\nu} = 0 \quad \text{and} \quad \pdv{\mathcal{L}}{(\partial_\mu A_\nu)} = - \partial^\nu A^\nu + (\partial_\rho A^\rho)\eta^{\mu\nu}
	\]
	we get the actual form of the equations of motion:
	\begin{equation}
		\partial_\mu F^{\mu\nu} = 0.
	\end{equation}

	\subsection{Symmetries}
	(For more, see the course \emph{Symmetries, Fields and Particles}.)
	\begin{defi}
		A \emph{symmetry} of some theory is a variation of fields which leaves the action invariant.
	\end{defi}

	Why are symmetries important? According to \emph{Noether's theorem}, symmetries result in certain conservation laws in a theory. The possible form of Lagrangian is restricted by the symmetries the theory has.

	We now discuss certain types of symmetries.

	\subsubsection{Spacetime Symmetry}
	Typical spacetime symmetries include
	\begin{itemize}
		\item Translation: $x^\mu \to x^\mu + c^\mu$ where $c^\mu \in \mathbb{R}^{3,1}$. Under translation, a scalar field transforms as $\phi(x) \to \phi(x-c)$; 
		\item Lorentz transformation: $x^\mu \to \LT{\mu}{\nu}x^\nu$ and $\phi(x) \to \phi(\Lambda^{-1} \cdot x)$;
		\item Scale transformation: $x^\mu \to \lambda x^\mu$ where $\lambda \in \mathbb{R}^+$ and the field goes as $\phi(x) \to \lambda^{-\Delta}\phi(\lambda^{-1} x)$ where $\Delta = [\phi]$ (this is for massless theories). 
	\end{itemize}

	\subsubsection{Internal Symmetry}

	For example, electric charge, flavour and colour arise from certain internal symmetries.

	\begin{ex}
		For a complex scalar field $\psi : \mathbb{R}^{3,1}\to \mathbb{C}$, the Lagrangian is
		\[
			\partial_\mu \psi^* \partial^\mu \psi - V(\abs{\psi}^2).
		\]
		 
		This theory has a symmetry 
		\[
			\psi(x) \to \psi'(x) = e^{\mathrm{i}\alpha}\psi(x) \quad \text{and} \quad \psi^*(x) \to \psi^*\/'(x) = e^{-\mathrm{i}\alpha}\psi^*(x)
		\]
		where $\alpha \in [0,2\pi)$. This leaves both $\mathcal{L}$ and $S$ invariant.
	\end{ex}

	Continuous symmetries form matrix Lie groups.
	\begin{ex}
		Lorentz transformation $x^\mu \to \LT{\mu}{\nu} x^\nu$. Note $\Lambda$ is a $4 \times 4$ matrix. Then the Lorentz group is
		\[
			G_L = \{\Lambda \in \Mat_4(\mathbb{R}): \Lambda \eta \Lambda^T = \eta, \det \Lambda = 1\} = \mathrm{SO}(3,1).
		\]
		
		The general picture is
		\[
			\underbrace{\mathrm{SO}(3,1)}_{\text{Lorentz}} \subset \underbrace{\text{Poincar\'e}}_{\text{Lorentz + translations}} \subset \underbrace{\mathrm{SO}(4,2)}_{\text{Conformal}}.
		\]
	\end{ex}

	\subsection{Finite vs Infinitesimal Transformations}

	\begin{defi}
		A (matrix) group element $g \in G$ are said to be \emph{near identity} if it can be written as 
		\[
			g = \exp (\alpha X):= \sum_{n=0}^\infty \frac{1}{n!} (\alpha X)^n\ \stackrel{\alpha \ll 1}{\simeq}\ \1 + \alpha X + \mathcal{O}(\alpha^2), \quad \alpha \in \mathbb{R}
		\]
		where the matrix $X \in \mathbb{L}(G)$, the Lie algebra of $G$.  
	\end{defi}

	It is often convenient to work to first order, e.g.
	\begin{ex}
		For a theory with
		\[
			\mathcal{L} = \partial_\mu \psi^* \partial^\mu \psi - V(\abs{\psi}^2)
		\]

		A finite transformation is $\psi(x) \to g \cdot \psi(x)$ with $g = \exp(\mathrm{i} \alpha) \in G \simeq \mathrm{U}(1)$.
		
		An infinitesimal transformation is $ \psi(x) \to \psi(x) + \delta\psi (x)$ with $\delta \psi(x) = \mathrm{i} \alpha \psi(x)$.
		
	\end{ex}

	\lecnr{4}
	\begin{ex}
		Consider proper Lorentz transformations
		\[
			x^\mu \to x'^\mu = \LT{\mu}{\nu} x ^{\nu}
		\]
		with
		\[
			\LT{\mu}{\alpha} \eta ^{\alpha \beta} \LT{\nu}{\beta} = \eta ^{\mu \nu} \quad \text{and} \quad \det (\Lambda) = +1.
		\]
		
		Such infinitesimal transformations are of the form \[
			\Lambda \approx \exp(s \Omega)
		\]
		where $\Omega \in \mathbb{L}(\SO(3,1))$. If we expand $\Lambda$ near identity, we have
		\[
			\LT{\mu}{\nu} = \delta^\mu _\nu + s \Omega\indices{^{\mu}_{\nu}} + \mathcal{O}(s^2)
		\]
		and the condition that $\Lambda$ satisfies gives to first order
		\[
			(\delta^\mu_\alpha + s \Omega\indices{^\mu_\alpha})\eta ^{\alpha \beta} (\delta^\nu_\beta + s \Omega\indices{^\nu _\beta}) = \eta ^{\mu \nu}.
		\]
		
		The term linear in $s$ gives
		\[
			\Omega ^{\mu \nu} + \Omega ^{\nu \mu} = 0 \quad \Rightarrow \quad \Omega _{\mu \nu} = - \Omega _{\nu \mu}.
		\]
		and the number of free entries in $\Omega$ is $\frac{1}{2} \times 3 \times 4 = 6 =$ 3 rotations + 3 boosts.

		Scalar field transforms as
		\[
			\phi(x) \to \phi(\Lambda^{-1}\cdot x)
		\]
		and from
		\[
			(\Lambda^{-1})\indices{^\mu_\nu} = \delta^\mu_\nu - s \Omega\indices{^\mu_\nu} + \mathcal{O}(s^2)
		\]
		we have
		\[
			\phi(\Lambda^{-1} \cdot x) \approx \phi(x - s \Omega \cdot x) \approx \phi(x) - s \Omega\indices{^\mu_\nu} x^\nu \partial_\mu \phi(x) + \mathcal{O}(s^2).
		\]
		
		Thus we conclude that the infinitesimal Lorentz transformation is of the form
		\[
			\phi(x) \to \phi(x) + \delta \phi(x)
		\]
		and 
		\begin{equation}
			\delta \phi(x) = -s \Omega\indices{^\mu_\nu} x^\nu \partial_\mu \phi(x).
		\end{equation}
	\end{ex}

	\subsection{More General Story}

	We can consider the more general case of complex fields taking value in $\mathbb{C}$
	\[
		\bm{\psi}: \mathbb{R}^{3,1} \to \mathbb{C}^n
	\]
	with inner product
	\[
		(\bm{\psi}_1, \bm \psi_2) = \bm \psi_1^\dagger \cdot \bm \psi_2.
	\]
	
	To work with such ``vector-like'' fields, we need the following definition.

	\begin{defi}
		An \emph{$n$-dimensional representation} of symmetry group $G$ is a map
		\[
			D : G \to \Mat_n(\mathbb{C})
		\]
		that preserves the group multiplication,
		\[
			D(g_1 \cdot g_2) = D(g_1)\cdot D(g_2), \quad \forall g_1, g_2 \in G.
		\]

		A representation $D$ of $G$ is \emph{unitary} if
		\begin{equation}
			D(g)^{-1} = D(g)^\dagger, \quad \forall g \in G.
			\label{eq:1.6.1}
		\end{equation}
	\end{defi}

	Given such a unitary $n$-dimensional representation of a symmetry group $G$, we can write down a Lagrangian that has a symmetry corresponding to this.

	\begin{equation*}
		\mathcal{L} = (\partial_\mu \bm \psi, \partial^\mu \bm \psi) - V((\bm \psi, \bm \psi))
	\end{equation*}
	is invariant under the symmetry
	\begin{equation*}
		\bm \psi (x) \to \bm \psi'(x) = D(g) \cdot \bm \psi(x), \quad \forall g \in G.
	\end{equation*}

	We can check that
	\[
		(\bm \psi, \bm \psi) \to (D(g) \cdot \bm \psi, D(g) \cdot \bm \psi) = (D(g)^\dagger \cdot D(g) \cdot \bm \psi, \bm \psi) \overset{(\ref{eq:1.6.1})}{=} (\bm \psi,\bm \psi).
	\]
	
	For such representations, the simplest possibility is that $G = \U(n)$, the group of $n \times n$ unitary matrices. For this, we have the \emph{fundamental representation}
	\[
		D_F(g) = g, \quad \forall g \in G.
	\]
	
	\subsection{Noether's Theorem}

	The general idea of Noether's theorem is that
	\[
		\text{continuous symmetry} \quad \Rightarrow \quad \text{conserved current}.
	\]

	\subsubsection{Continuous Symmetry}

	Consider a continuous symmetry of a scalar field
	\[
		\phi(x) \to \phi(x) + \delta \phi(x) \quad \text{with} \quad \delta \phi(x) = X(\phi(x), \partial \phi(x), \cdots).
	\]
	
	\begin{ex}
		\
		\begin{itemize}
			\item Lorentz transformation $\Lambda = \exp(s \Omega)$ has \[
				X = - s \Omega\indices{^\mu _\nu} x^\nu \partial_\mu \phi(x);
			\]
			\item Translation $\phi(x) \to \phi(x - \varepsilon)$ has \[
				X = - \varepsilon^\mu \partial_\mu \phi(x).
			\]
		\end{itemize}
	\end{ex}

	Now investigate the variation of $\mathcal{L}(x)$. 
	\begin{itemize}
		\item For Lorentz transformation, since $\mathcal{L}(x)$ is a scalar field, we have \[
			\delta \mathcal{L}(x) = -s \Omega\indices{^\mu _\nu} x^\nu \partial_\mu \mathcal{L}(x) = s \Omega\indices{^\mu _\nu} \partial_\mu (x^\nu \mathcal{L}(x)) = \text{total derivative}
		\]
		using 
		\[
			\Omega\indices{^\mu_\mu} = \eta ^{\mu \nu} \Omega _{\mu \nu} = 0.
		\]
		\item And for translation $\mathcal{L}(x) \to \mathcal{L}(x - \varepsilon)$ \[
			\delta \mathcal{L}(x) = - \varepsilon^\mu \partial_\mu \mathcal{L}(x) = \text{total derivative}.
		\]
	\end{itemize}

	For a general symmetry, we can assume
	\begin{equation}
		\delta \mathcal{L}(x) = \partial_\mu F^\mu
		\label{eq:1.7.1}
	\end{equation}
	for some $F^\mu = F^\mu(\phi(x), \partial \phi(x), \cdots)$.
	
	The variation of action is
	\[
		\delta S = \int _{\mathbb{R}^{3,1}} \dd[4]{x} \delta \mathcal{L} = \int _{\mathbb{R}^{3,1}} \dd[4]{x} \partial_\mu F^\mu \overset{\text{Stokes}}{=} \int _{\partial(\mathbb{R}^{3,1})} \dd{S_\mu} F^\mu = \text{surface term}
	\]
	but we always assume the action is not affected by any surface-term-like variations.

	\subsubsection{Conserved Current}

	\begin{defi}
		\emph{Conserved current} is a 4-vector field
		\[
			j^\mu (x) = j^\mu(\phi(x), \partial \phi(x),\cdots)
		\]
		such that
		\begin{equation}
			\partial_\mu j^\mu = 0
			\label{eq:1.7.2}
		\end{equation}
		when $\phi$ obeys its Euler-Lagrange equation(s).

		We can further define
		\[
			j^\mu (x) = j^\mu (t, \vb x) = (j^0 (t,\vb x), \vb J(t, \vb x))
		\]
		where we call $j^0$ the \emph{charge density} and $\vb J$ the \emph{current density}. The \emph{total charge} in any region of space $V \subset \mathbb{R}^3$ is defined as
		\[
			Q_V (t) = \int_V \dd[3]{\vb x} j^0(t, \vb x).
		\]
	\end{defi}
	
	Recall the condition (\ref{eq:1.7.2}) can be re-written as
	\begin{equation}
		\pdv{j^0}{t}(t,\vb x) + \div \vb J (t, \vb x) = 0
		\label{eq:1.7.3}
	\end{equation}
	and we have
	\[
		\dv{Q_V(t)}{t} = \int_V \dd[3]{\vb x} \pdv{t} j^0(t, \vb x) \overset{(\ref{eq:1.7.3})}{=} - \int_V \dd[3]{\vb x} \div \vb J(t, \vb x) \overset{\text{Stokes}}{=} - \int _{\partial V} \dd{\vb S} \vdot \vb J (t, \vb x)
	\]
	which is the flux through $\partial V$. When $V = \mathbb{R}^3$ and $\phi(x)\to 0$ as $\abs{\vb x}\to \infty$, we have the total charge of whole $\mathbb{R}^3$
	\[
		Q(t) = \int _{\mathbb{R}^3} \dd[3]{\vb x} j^0(t, \vb x)
	\]
	and 
	\[
		\dv{Q(t)}{t} = - \int _{\partial \mathbb{R}^3} \dd{\vb S} \vdot \vb J(t, \vb x) \overset{\text{b.c.}}{=} 0.
	\]

	\subsubsection{Proof of the Theorem}

	\begin{thm}[Noether's theorem]
		A (continuous) symmetry gives rise to conserved current(s).
	\end{thm}
	\begin{proof}
		Consider the variation of the Lagrangian
		\begin{align*}
			\delta \mathcal{L}(x) = &\ \pdv{\mathcal{L}}{\phi} \delta \phi + \pdv{\mathcal{L}}{(\partial_\mu \phi)} \delta(\partial_\mu \phi) \\
			\overset{\text{Leibniz}}{=} &\ \pdv{\mathcal{L}}{\phi} \delta \phi - \partial_\mu \left( \pdv{\mathcal{L}}{(\partial_\mu \phi)} \right) \delta \phi + \partial_\mu \left( \pdv{\mathcal{L}}{(\partial_\mu \phi)} \delta \phi \right)\\
			= &\ \underbrace{\left[ \pdv{\mathcal{L}}{\phi} - \partial_\mu \left( \pdv{\mathcal{L}}{(\partial_\mu \phi)} \right) \right]}_{0 \text{ by E-L}} \delta \phi + \partial_\mu \left( \pdv{\mathcal{L}}{(\partial_\mu \phi)} \delta \phi \right).
		\end{align*}
		thus
		\begin{equation}
			\delta \mathcal{L} = \partial_\mu \left( \pdv{\mathcal{L}}{(\partial_\mu \phi)} X(\phi) \right)
			\label{eq:1.7.4}
		\end{equation}

		Define
		\[
			j^\mu (x) := \pdv{\mathcal{L}}{(\partial_\mu \phi)} X(\phi(x)) - F^\mu(\phi(x)).
		\]
		
		Compare (\ref{eq:1.7.1}) and (\ref{eq:1.7.4}) we get
		\begin{equation}
			\partial_\mu j^\mu = \partial_\mu \left( \pdv{\mathcal{L}}{(\partial_\mu \phi)} X(\phi(x)) \right) - \partial_\mu F^\mu (\phi(x)) = 0.
		\end{equation}
	\end{proof}
	
	\subsubsection{Application of NT: Translational Symmetry} \lecnr{5}

	For a translation $x^\mu \to x^\mu + \varepsilon^\mu$ with small $\varepsilon$, we expand the field 
	\[
		\phi(x) \to \phi(x - \varepsilon) \approx \phi(x) - \varepsilon^\mu \partial_\mu \phi(x) + \mathcal{O}(\varepsilon^2).
	\]
	
	Infinitesimally, we identify 
	\[
		\delta \phi(x) = - \varepsilon^\mu \partial_\mu \phi(x).
	\]
	
	The Lagrangian itself is a scalar field, thus 
	\[
		\delta \mathcal{L}(x) = - \varepsilon^\mu \partial_\mu \mathcal{L}(x)
	\]
	as we mentioned before. Following the proof of Noether's theorem, we can construct 
	\[
		\delta \phi(x) = \varepsilon^\nu X_\nu (\phi) \quad \text{and} \quad \delta \mathcal{L} (x) = \varepsilon^\nu \partial_\mu F\indices{^\mu_\nu}(\phi)
	\]
	and identify
	\[
		X_\nu = \partial_\nu \phi \quad \text{and} \quad F\indices{^\mu_\nu} = \delta^\mu_\nu \mathcal{L}.
	\]
	
	Noether's theorem implies we have a 4-parameter family of conserved currents, which can be expressed in terms of the \emph{energy-momentum tensor}
	\begin{equation}
		T\indices{^\mu_\nu} = (j^\mu)_\nu := \pdv{\mathcal{L}}{(\partial_\mu \phi)} \partial_\nu \phi - \delta^\mu_\nu \mathcal{L}
		\label{eq:1.7.6}
	\end{equation}
	with conservation law
	\[
		\partial_\mu T\indices{^\mu_\nu} = 0
	\]
	for free index $\nu = 0,1,2,3$.

	The corresponding conserved charges are the \emph{energy}
	\[
		E = \int _{\mathbb{R}^3} \dd[3]{\vb x} T ^{00}
	\]
	for $\nu = 0$, and \emph{3-momentum}
	\[
		(\vb P)^i = \int _{\mathbb{R}^3} \dd[3]{\vb x} T ^{0i}
	\]
	for $i = \nu = 1,2,3$.

	We can combine these into a \emph{4-momentum} $P^\nu = (E, \vb P)$ and
	\[
		P^\nu = \int _{\mathbb{R}^3} \dd[3]{\vb x} T ^{0 \nu}.
	\]
	
	\begin{ex}[Klein-Gordon theory]
		K-G theory has Lagrangian
		\[
			\mathcal{L} = \frac{1}{2} \partial_\mu \phi \partial^\mu \phi - \frac{1}{2} m^2 \phi^2.
		\]
		
		Using $\pdv{\mathcal{L}}{(\partial_\mu \phi)} = \partial^\mu \phi$, we have the energy-momentum tensor
		\[
			T ^{\mu \nu} = \partial^\mu \phi \partial^\nu \phi - \eta ^{\mu \nu} \mathcal{L} = \partial^\mu \phi \partial^\nu \phi - \frac{1}{2}\eta ^{\mu \nu} \left( \partial_\rho \phi \partial^\rho \phi -  m^2 \phi^2 \right)
		\]
		
		Taking $\mu = \nu = 0$ we have the energy
		\[
			E = \int _{\mathbb{R}^3} \dd[3]{\vb x} \frac{1}{2} \left(\dot{\phi}^2 + \abs{\grad \phi}^2 +  m^2 \phi^2 \right)  \geq 0.
		\]
	\end{ex}

	\begin{nt}
		The above example gives a symmetric energy-momentum tensor (when two indices are all upstairs or downstairs). However, in general it may not be the case. One can add well-tuned total derivatives to make it symmetric.
	\end{nt}

	\subsection{Coupling to Gravity}

	For a flat space, the metric is $g _{\mu \nu} = \eta _{\mu \nu}$. The action is
	\[
		S = \int _{\mathbb{R}^{3,1}} \dd[4]{x} \mathcal{L}(x)
	\]
	with Lagrangian
	\[
		\mathcal{L} = \frac{1}{2} \partial_\mu \phi \partial^\mu \phi - V(\phi).
	\]
	
	In curved spacetime, we have a general metric $g _{\mu \nu} = g _{\mu \nu}(x)$. To describe the physics correctly, we need to change all partial derivatives into \emph{covariant derivatives}
	\[
		\partial_\mu \phi \to \nabla_\mu \phi.
	\]
	
	The volume integral is generalised as $\dd[4]{x} \to \dd[4]{x} \sqrt{-g}$ with $g = \det g _{\mu \nu}$ so the action now becomes
	\[
		\tilde{S} = \int _{\mathcal{M}} \dd[4]{x} \sqrt{-g} \tilde{\mathcal{L}} (x)
	\]
	where the Lagrangian itself becomes
	\[
		\tilde{\mathcal{L}} = \frac{1}{2} g ^{\mu \nu}(x) \nabla_\mu \phi \nabla_\nu \phi - V(\phi).
	\]

	From \emph{General Relativity}, the energy-momentum tensor $T _{\mu \nu}$ appears on the right-hand side of Einstein equation
	\[
		R _{\mu \nu} - \frac{1}{2} g _{\mu \nu} R = 8 \pi G T _{\mu \nu},
	\]
	which is the equation of motion from the sum of Einstein-Hilbert action $S _{\text{EH}}$ and the matter action $\tilde{S}$
	\[
		S _{\text{EH}} + \tilde{S}= \frac{1}{16 \pi G} \int \dd[4]{x} \sqrt{-g} R + \int \dd[4]{x} \sqrt{-g} \tilde{\mathcal{L}} (x).
	\]

	In particular, for Minkowski space
	\[
		T _{\mu \nu}(x) = - \frac{2}{\sqrt{-g}} \pdv{(\sqrt{-g} \tilde{\mathcal{L}})}{g ^{\mu\nu}}\bigg|_{g_{\mu \nu} = \eta_{\mu \nu}},
	\]
	i.e.\ we can use the general matter-coupling theory and evaluate at Minkowski metric.

	\newpage

	\section{Canonical Quantisation}

	\subsection{Hamiltonian Formulation}
	
	Recall in \emph{Quantum Mechanics}, we canonically quantised classical mechanics in its Hamiltonian formulation. To use such canonical quantisation, we need to reformulate our (classical) field theory in terms of \emph{Hamiltonian} first.

	\subsubsection{Review of Hamiltonian Formulation}

	Consider a non-relativistic particle (with unit mass, WLOG), moving in one dimension. It has position coordinate
	\[
		q = q(t)
	\]
	
	If it is put in a potential $V = V(q)$, then easily its Lagrangian is
	\[
		L (q(t), \dot{q}(t)) = \frac{1}{2} \dot{q}^2 - V(q).
	\]
	
	\begin{defi}
		The \emph{momentum} $p$ \emph{conjugate to} $q$ is defined to be 
		\begin{equation}
			p := \pdv{L}{\dot{q}}.
			\label{eq:2.1.1}
		\end{equation}
	\end{defi}

	In this simple case, $p = \dot q$.
	
	\begin{defi}
		We can define the \emph{Hamiltonian} $H$ by performing \emph{Legendre transform}
		\begin{equation}
			H := p \dot q - L(q,\dot q).
			\label{eq:2.1.2}
		\end{equation}
	\end{defi}

	In this case, we can eliminate $\dot q$ to give the Hamiltonian
	\[
		H = H(p,q) = \frac{1}{2} p^2 + V(q) = \text{total energy}.
	\]
	
	Now consider the Hamiltonian mechanics for $N$ degrees of freedom. The Lagrangian now depends on $N$ coordinates and their time derivatives
	\[
		L = L \left( \{q_i(t)\}, \{\dot{q}_i(t)\} \right)
	\]
	for $i = 1,\cdots, N$.

	Each coordinate $q_i$ has a conjugate momentum associated
	\begin{equation}
		p_i := \pdv{L}{\dot{q}_i}.
		\label{eq:2.1.3}
	\end{equation}
	
	\begin{defi}
		The \emph{Poisson bracket} on functions $F = F(p,q)$ and $G = G(p,q)$ is defined as
		\begin{equation}
			\{F,G\} := \sum _{i = 1}^{N} \pdv{F}{q_i} \pdv{G}{p_i} - \pdv{F}{p_i} \pdv{G}{q_i}.
			\label{eq:2.1.4}
		\end{equation}
	\end{defi}

	\begin{nt}
		The Poisson bracket acts like a commutator.
	\end{nt}

	(\ref{eq:2.1.4}) gives 
	\begin{equation}
		\{q_i, q_j\} = \{p_i, p_j\} = 0 \quad \text{and} \quad \{q_i, p_j\} = \delta _{ij}.
		\label{eq:2.1.5}
	\end{equation}

	Now we generalise the Hamiltonian for $N$ degrees of freedom
	\begin{equation*}
		H = H(\{q_i\}, \{p_i\}) := \sum _{i=1} ^{N} p_i \dot{q}_i - L (\{q_i\}, \{\dot{q}_i\})
	\end{equation*} 
	
	Eliminate the time derivatives $\dot{q}_i$ via equation (\ref{eq:2.1.3}) so that we have
	\[
		H = H(\{p_i\},\{q_i\}).
	\]
	
	The initial state of the system (at $t=0$) is given by specifying the points
	\[
		q_i(0), \quad p_i(0), \quad i = 1, \cdots, N.
	\]
	in \emph{phase space} $\{(q_i,p_i)\}$.
	
	The time evolution of the system is generated by Hamiltonian. For any function $F(p,q)$ we have its time derivative satisfying

	\begin{prop}
		\[
			\dot{F} = \{F,H\},
		\]
		known as the \emph{Hamilton's equation}.
	\end{prop}
	
	In particular, if $F= p_i$ or $F = q_i$, we get the \emph{Hamilton's equations of motion}
	\begin{align*}
		\dot{q}_i & = \{q_i, H\} = \pdv{H}{p_i}\\
		\dot{p}_i & = \{p_i, H\} = - \pdv{H}{q_i}.
	\end{align*}

	Suppose we have $F = Q(p,q)$ a conserved quantity, then we have 
	\[
		\dot{Q} = 0 \quad \Rightarrow \quad \{Q,H\} = 0.
	\]
	
	\subsubsection{Quantisation of Classical Mechanics} \lecnr{6}

	To quantise the classical theory, now the \emph{states} of the system correspond to vectors (rays) in some Hilbert space $\mathcal{H}$,
	\[
		\ket{\psi}\in \mathcal{H}.
	\]
	
	The \emph{observables} of the system are promoted from functions of phase space to some operator
	\[
		F(p,q) \quad \xrightarrow{\text{quantisation}} \quad \hat{F}: \mathcal{H} \to \mathcal{H}.
	\]

	The Poisson bracket of the classical theory now becomes commutator of operators
	\[
		\{\cdot,\cdot\} \quad \xrightarrow{\text{quantisation}} \quad \frac{1}{\mathrm{i} \hbar} [\cdot, \cdot].
	\]
	
	In particular, the coordinates and momenta become
	\begin{align*}
		q_i \quad \to \quad \hat{q}_i,\\
		p_i \quad \to \quad \hat{p}_i.
	\end{align*}
	
	The rule (\ref{eq:2.1.5}) now becomes the commutation relations
	\begin{equation}
		[\hat{q}_i, \hat{q}_j] = [\hat{p}_i, \hat{p}_j] = 0 \quad \text{and} \quad [\hat{q}_i, \hat{p}_j] = \mathrm{i} \hbar \delta _{ij}.
		\label{eq:2.1.6}
	\end{equation}

	\begin{nt}
		In general, when quantising $F(p,q)$ to $\hat{F}(\hat p, \hat q)$, if there is any product of $\hat p, \hat q$ present, there would be \emph{ordering ambiguities}. One need to specify more information to resolve this. We will not concern about this for now.
	\end{nt}

	Time evolution is generated by $\hat H(\hat p, \hat q)$. Now take \emph{Schr\"odinger picture}, i.e.\ taking states to be time-dependent, we have
	\[
		- \mathrm{i} \hbar \pdv{t} \ket{\psi(t)} = \hat H \ket{\psi(t)}.
	\]
	
	The energy eigenstates of the system are eigenstates of $\hat H$,
	\[
		\hat H \ket{\psi} = E \ket{\psi}
	\]
	with energy (eigenvalues) $E$.
	
	\subsubsection{Hamiltonian Formulation of Field Theory}

	\emph{Field theory} is an infinite-dimensional dynamical system. Recall in our route from classical (particle) dynamics to classical field theory, we went from discrete labels (with $N$ d.o.f.) to a continuous label (with infinite d.o.f)
	\[
		\underbrace{q_i(t)}_{\substack{\text{discrete labels}\\i = 1,\cdots,N}} \quad \longrightarrow \quad \underbrace{\phi(t, \vb x)}_{\substack{\text{continuous label}\\\vb x \in \mathbb{R}^3}}.
	\]

	\begin{nt}
		There are two types of infinity involved here:
		\begin{itemize}
			\item \emph{Infrared} (IR): infinite range of $\vb x$;
			\item \emph{Ultraviolet} (UV): continuous infinity.
		\end{itemize}
	\end{nt}

	To quantise classical field theory, we first reformulate it in terms of Hamiltonian.

	\begin{defi}
		For a field $\phi(t, \vb x)$, its \emph{momentum $\pi(t, \vb x)$ conjugate to $\phi(t, \vb x)$} is defined to be
		\begin{equation}
			\pi (t, \vb x) = \pdv{\mathcal{L}(t,\vb x)}{\dot \phi(t,\vb x)}.
		\end{equation}
	\end{defi}
	
	\begin{defi}
		The \emph{Hamiltonian density} $\mathcal{H}(t,\vb x)$ of $\phi(t,\vb x)$ is defined through Legendre transform of Lagrangian (density),
		\begin{equation}
			\mathcal{H}(t,\vb x) = \pi(t,\vb x) \dot \phi(t,\vb x) - \mathcal{L}(t,\vb x).
			\label{eq:2.1.8}
		\end{equation}
	\end{defi}
	\begin{nt}
		The time derivative dependence is eliminated by (\ref{eq:2.1.8}) in the Hamiltonian density
		\[
			\mathcal{H} = \mathcal{H}(\phi(t,\vb x), \pi(t,\vb x)).
		\]
	\end{nt}

	\begin{defi}
		The \emph{Hamiltonian} is defined by
		\begin{equation*}
			H = \int _{\mathbb{R}^3} \dd[3]{\vb x} \mathcal{H}(t, \vb x).
		\end{equation*}
	\end{defi}

	\begin{ex}
		For 
		\[
			\mathcal{L} = \frac{1}{2} \partial_\mu \phi \partial^\mu \phi - V(\phi) = \frac{1}{2} \dot \phi ^2 - \frac{1}{2} \abs{\grad \phi}^2 - V(\phi),
		\]
		the conjugate momentum is
		\[
			\pi(t,\vb x) = \pdv{\mathcal{L}}{\dot \phi(t,\vb x)} = \dot \phi(t, \vb x).
		\]

		The Hamiltonian density is
		\[
			\mathcal{H} = \pi \dot \phi - \mathcal{L} = \frac{1}{2} \pi^2 + \frac{1}{2} \abs{\grad \phi}^2 + V(\phi) {\color{red}\ = T ^{00}} .
		\]
		where $T ^{00}$ is the $00$-component of energy-momentum tensor calculated earlier.

		The Hamiltonian is
		\[
			H = \int _{\mathbb{R}^3} \dd[3]{\vb x} \left(  \frac{1}{2} \pi^2 + \frac{1}{2} \abs{\grad \phi}^2 + V(\phi) \right) {\color{red}\ = \int _{\mathbb{R}^3} \dd[3]{\vb x} T ^{00} = E}.
		\]
		where $E$ is the energy defined earlier.
	\end{ex}

	\subsection{Canonical Quantisation}

	\subsubsection{Quantisation}

	First, we generalise the Poisson bracket to classical fields by requiring
	\[
		\{\phi(t,\vb x), \pi(t, \vb y)\} = \delta ^{(3)}(\vb x - \vb y)
	\]
	at equal time $t$. We can abbreviate for fixed time
	\begin{align*}
		\phi(\vb x) & := \phi(t, \vb x)\\
		\pi(\vb x) & := \pi(t, \vb x).
	\end{align*}

	Now, to quantise, we promote fields to field operators
	\begin{align*}
		\phi(\vb x) \quad \xrightarrow{\text{quantisation}} \quad \hat \phi(\vb x),\\
		\pi(\vb x) \quad \xrightarrow{\text{quantisation}} \quad \hat \pi(\vb x).
	\end{align*}
	
	Similar to quantum mechanics, we require the commutators (at equal time) to satisfy
	\[
		[\hat \phi(\vb x), \hat \phi(\vb y)] = [\hat \pi(\vb x), \hat \pi(\vb y)] = 0 \quad \text{and} \quad [\hat \phi(\vb x), \hat \pi(\vb y)] = \mathrm{i} \delta ^{(3)}(\vb x - \vb y).
	\]
	where we've set $\hbar = 1$ (and throughout this course).

	This is the \emph{Canonical Quantisation}.

	Hamiltonian naturally becomes
	\[
		\hat H = \int _{\mathbb{R}^3} \dd[3]{\vb x} \left( \frac{1}{2} \hat \pi^2 + \frac{1}{2} \abs{\grad \hat \phi (\vb x)}^2 + V(\hat \phi) \right).
	\]
	
	\subsubsection{States (Heuristic Discussion)}

	Consider eigenstates $\ket{f}$ of the field operator $\hat \phi(\vb x)$ that
	\[
		\hat \phi(\vb x) \ket{f} = f(\vb x)\ket{f}
	\]
	where the eigenvalues are functions
	\[
		f: \mathbb{R}^3 \to \mathbb{R}.
	\]
	
	If work in basis of these eigenvectors, these should span Hilbert space
	\[
		\mathcal{H} \overset{?}{=} \Span _{\mathbb{C}} \{\ket{f} |\ f : \mathbb{R}^3 \to \mathbb{R}\}.
	\]

	Think about a generic state $\ket{\psi}\in \mathcal{H}$, it corresponds to a \emph{wave-functional}
	\[
		\Psi[f(\vb x)] = \braket{f}{\psi} \in \mathbb{C}.
	\]
	
	The Hamiltonian becomes a Hermitian operator
	\[
		\hat{\mathbb{H}}: \mathcal{H}\to \mathcal{H}.
	\]
	
	If we would like to find the energy eigenstate, we need to solve the functional Schr\"odinger equation
	\[
		\hat{\mathbb{H}} \cpf \Psi = E \Psi
	\]
	which is damn hard (nearly impossible in general) to solve.
\end{document}