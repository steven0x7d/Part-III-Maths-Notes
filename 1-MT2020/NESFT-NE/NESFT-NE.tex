\documentclass[a4paper,11pt]{article}

\usepackage{../../zyancamlec}

\def\ntripos{Mathematical Tripos}
\def\npart{III Non-Examinable}
\def\ncourse{Non-Equilibrium Statistical Field Theory}
\def\nscourse{NESFT}
\def\nlecturer{J.\ Pausch}
\def\nterm{Michaelmas}
\def\nyear{2020}

\begin{document}
	\maketitlepage
	\preliminaries

	\section*{Introduction}
	
	\noindent This course introduces Master Equations --- an ODE for time dependent probability distributions --- as a model for reaction systems on lattices. After briefly exploring its basic properties and models such as random walkers as well as branching and coagulation processes, they are transformed into a second-quantized form using bosonic ladder operators and coherent states. The meaning and properties associated with shifts of the creation and annihilation operators are explored as well as the interpretation of observables. Then, the second quantized equation is formally solved in a path integral. At this point, Doi-Peliti field theory starts and a few example models are presented: diffusion, branching, coagulation as well as coupled reactions. Feynman diagrams, loop corrections, n-point correlation functions are all concepts which are introduced and Part III students taking SFT will hopefully enjoy a non-equilibrium point of view on them.

	\noindent Next, the lecture jumps back to pre-field-theory models and introduces Langevin Equations with a few example processes such as the Ornstein-Uhlenbeck process. This is done in order to derive a different field theory, called Response field formalism, based on work by Martin, Siggia, Rose, Janssen and DeDominicis. The previously described examples are now revisited as field theories and, if time allows, Doi-Peliti field theory is brought back to be combined with Response Field theories in the role of prior distributions.

	\section*{Pre-requisites}

	It will be very beneficial to take the Part III course \emph{Statistical Field Theory} (SFT) alongside this course. Although both courses talk about different statistical field theories, they explore many of the same concepts.

	\newpage
	\section*{Overview}
	\begin{enumerate}
		\item Master Equations
		\item Second Quantisation
		\item Doi-Peliti Field Theory
		\item Langevin Equation
		\item Response-Field Formalism
	\end{enumerate}

	Q: What does \emph{non-equilibrium} mean? 
	
	A: It means we consider the statistical field theory under \emph{stochastic processes}.

	Q: What does \emph{field} mean? 
	
	A: It means an approach using \emph{path integrals} or say, \emph{Feynman diagrams}.
	\newpage
	\tableofcontents
	\newpage
	\maintext

	\section{Master Equations}

	Master equations encloses how particles move in time and space on a statistical basis.

	\subsection{Derivation}

	We use discrete steps in time: $t \in \Delta t \mathbb{Z}$. The convention for indices follows 
	$$\cdots > t_3 > t_2 > t_1 > \cdots$$

	Note that strangely, we put later times to the left. (Using conditions from probability theory, etc.)

	We regard particle numbers $N(t) \in \mathbb{N}_0$ are random variables.

	The conditional probability
	\[
		P\left(N(t) = n | N(s) = m\right) = \frac{P\left(N(t) = n , N(s) = m\right)}{P\left(N(s) = m\right)}
	\]
	
	Markov property
	\[
		P(N(t_{n+1})| N(t_n), \cdots, N(t_1)) = P(N(t_{n+1})|N(t_n))
	\]
	As known as ``memoryless'' stochastic property.

	\paragraph{\underline{Chapman-Kolmogorov Equation}} 
	\ 

	By the definition of conditional probability, we can compute
	\[
		P(N(t_3), N(t_2), N(t_1)) = P(N(t_3)| N(t_2), N(t_1))\ P(N(t_2)|N(t_1))\ P(N(t_1))
	\]
	
	If we sum all possible intermediate particle numbers, we have the C-K equation
	\begin{equation}
		\boxed{P(N(t_3)| N(t_1)) = \sum_{N(t_2)} P(N(t_3)|N(t_2))\ P(N(t_2)|N(t_1))}
	\end{equation}
	
	The above equation can be seen as sum over all possible paths between $(t_1, N(t_1))$ and $(t_3, N(t_3))$ in the space $\{t, N(t)\}$.

	We may have multiple intermediate points:
	\begin{equation}
		P(N(t_4)| N(t_1)) = \sum_{N(t_3)} \sum_{N(t_2)} P(N(t_4)|N(t_3))\ P(N(t_3)|N(t_2))\ P(N(t_2)|N(t_1))
	\end{equation}

	If we could extend this to a continuous limit in time to get a \emph{path integral}. To show this, we first define
	\[
		W_t (N'|N) = \pdv{P(N'(t')|N(t))}{t'}\Bigg|_{t' = t}
	\]

	Then if we Taylor expand
	\[
		P(N'(t + \Delta t)|N(t)) = \underbrace{P(N'(t)|N(t))}_{\delta_{N'(t),N(t)}} + \Delta t\ W_t(N'|N) + \mathcal{O}(\Delta t^2)
	\]
	where the $\delta$ is Kronecker delta.

	We can re-write the C-K equation as 
	\[
		P(N(t_3)| N(t_1)) = \sum_{N(t_2)} \left\{\underbrace{\delta_{N_3,N_2}}_{1} + \Delta t\ W_{t_2}(N(t_3)|N(t_2)) \right\} P(N(t_2)|N(t_1))
	\]
	and rearrange
	\[
		P(N(t_3)|N(t_1)) - P(N(t_2)|N(t_1))\big|_{t_2 = t_3} = \sum_{N(t_2)} \Delta t\ W_{t_2}(N(t_3)|N(t_2))  P(N(t_2)|N(t_1))
	\]
	
	Note that
	\[
		\sum_{N'} P(N'|N) = 1 \quad \Rightarrow\quad \sum_{N'} W_t (N' | N) = 0 \quad \Rightarrow\quad W_t(N|N) = - \sum_{N' \neq N} W_t (N' | N)
	\]
	
	Thus
	\[
		P(N(t_3)|N(t_1)) - P(N(t_2)|N(t_1)) = \Delta t \sum_{N_2 \neq N_3} \left\{ W_{t_2}(N_3|N_2) P (N_2 | N_1) - W_{t_3}(N_2 | N_3)P(N_3|N_1)\right\}
	\]

	Then the final result is
	\begin{equation}
		\pdv{P(N(t) | N(t_1))}{t} = \sum_{N' \neq N} \left\{ W_t(N | N')P(N'(t) | N(t_1)) - W_t (N' | N)P(N(t) | N(t_1))\right\}
	\end{equation}

	Or, we can write it as the \emph{master equation} compactly
	\begin{equation}
		\pdv{P(N)}{t} = \sum_{N' \neq N} \left\{ W_t (N | N')P(N') - W_t(N'|N)P(N)\right\}
	\end{equation}

	The interpretation of this equation is
	\[
		\pdv{P(N)}{t} = \sum_{N' \neq N} \left\{ \underbrace{W_t (N | N')P(N')}_{\text{gain}} - \underbrace{W_t(N'|N)P(N)}_{\text{loss}}\right\}
	\]

	\begin{ex}[Extinction]
		A system with particles of species $A$ is undergoing a reaction $A \to \varnothing$.

		The transition function of decreasing by one particle is \[
			W(N-1|N) = N \epsilon
		\]
		with reaction rate $\epsilon$.

		Plug this into the master equation, we get
		\[
			\pdv{P(N)}{t} = \underbrace{\epsilon (N+1) P(N+1)}_{\text{gain}} - \underbrace{\epsilon N P(N)}_{\text{loss}}
		\]
		
		This also gives the hint that there is only one way that the particles decrease --- one by one.
	\end{ex}
	
	
	
\end{document}