\documentclass[a4paper,11pt]{article}

\usepackage{../../zyancamlec}

\def\ntripos{Mathematical Tripos}
\def\npart{III Non-Examinable}
\def\ncourse{Non-Equilibrium Statistical Field Theory}
\def\nscourse{NESFT}
\def\nlecturer{J.\ Pausch}
\def\nterm{Michaelmas}
\def\nyear{2020}

\begin{document}
	\maketitlepage
	\preliminaries

	\section*{Introduction}
	
	\noindent This course introduces Master Equations --- an ODE for time dependent probability distributions --- as a model for reaction systems on lattices. After briefly exploring its basic properties and models such as random walkers as well as branching and coagulation processes, they are transformed into a second-quantized form using bosonic ladder operators and coherent states. The meaning and properties associated with shifts of the creation and annihilation operators are explored as well as the interpretation of observables. Then, the second quantized equation is formally solved in a path integral. At this point, Doi-Peliti field theory starts and a few example models are presented: diffusion, branching, coagulation as well as coupled reactions. Feynman diagrams, loop corrections, n-point correlation functions are all concepts which are introduced and Part III students taking SFT will hopefully enjoy a non-equilibrium point of view on them.

	\noindent Next, the lecture jumps back to pre-field-theory models and introduces Langevin Equations with a few example processes such as the Ornstein-Uhlenbeck process. This is done in order to derive a different field theory, called Response field formalism, based on work by Martin, Siggia, Rose, Janssen and DeDominicis. The previously described examples are now revisited as field theories and, if time allows, Doi-Peliti field theory is brought back to be combined with Response Field theories in the role of prior distributions.

	\section*{Pre-requisites}

	It will be very beneficial to take the Part III course \emph{Statistical Field Theory} (SFT) alongside this course. Although both courses talk about different statistical field theories, they explore many of the same concepts.

	\newpage
	\tableofcontents
	\newpage
	\maintext
	Hi

	
\end{document}