\documentclass[a4paper,11pt]{article}

\usepackage{../../zyancamlec}

\def\ntripos{Mathematical Tripos}
\def\npart{III}
\def\ncourse{General Relativity}
\def\nscourse{GR}
\def\nlecturer{H.\ Reall}
\def\nterm{Michaelmas}
\def\nyear{2020}

\begin{document}
	\maketitlepage
	\preliminaries

	\section*{Introduction}

	\noindent General Relativity is the theory of space-time and gravitation proposed by Einstein in 1915. It remains at the centre of theoretical physics research, with applications ranging from astrophysics to string theory. This course will introduce the theory using a modern, geometric, approach.
	
	\noindent This is a second course on General Relativity, albeit one that could just about be followed without prior exposure to the subject. The first half of the course will give an introduction to differential geometry, the mathematics that underlies curved spacetime. The second half of the course will discuss the physics of gravity.

	\section*{Pre-requisites}
	
	\noindent Familiarity with Newtonian gravity, special relativity, finite-dimensional vector spaces and the Euler-Lagrange equations is essential. Knowledge of the relativistic formulation of Maxwell’s equations is highly desirable.
	
	\noindent Most students attending this course have already taken an introductory course in General Relativity (e.g. the Part II course). If you have not studied GR before then you should read an introductory book (e.g. Hartle or Rindler) before attending this course. Certain topics usually covered in a first course, e.g.\ the solar system tests of GR, will not be covered in this course.

	\newpage
	\tableofcontents
	\newpage
	\maintext
	Hi
	
\end{document}