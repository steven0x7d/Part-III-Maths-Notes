\documentclass[a4paper,11pt]{article}

\usepackage{../../zyancamlec}

\def\ntripos{Mathematical Tripos}
\def\npart{III}
\def\ncourse{General Relativity}
\def\nscourse{GR}
\def\nlecturer{H.\ Reall}
\def\nterm{Michaelmas}
\def\nyear{2020}

\begin{document}
	\maketitlepage

	\preliminaries

	\section*{Course Information}

	\noindent General Relativity is the theory of space-time and gravitation proposed by Einstein in 1915. It remains at the centre of theoretical physics research, with applications ranging from astrophysics to string theory. This course will introduce the theory using a modern, geometric, approach.
	
	\noindent This is a second course on General Relativity, albeit one that could just about be followed without prior exposure to the subject. The first half of the course will give an introduction to differential geometry, the mathematics that underlies curved spacetime. The second half of the course will discuss the physics of gravity.

	\section*{Pre-requisites}
	
	\noindent Familiarity with Newtonian gravity, special relativity, finite-dimensional vector spaces and the Euler-Lagrange equations is essential. Knowledge of the relativistic formulation of Maxwell’s equations is highly desirable.
	
	\noindent Most students attending this course have already taken an introductory course in General Relativity (e.g. the Part II course). If you have not studied GR before then you should read an introductory book (e.g. Hartle or Rindler) before attending this course. Certain topics usually covered in a first course, e.g.\ the solar system tests of GR, will not be covered in this course.

	\newpage
	\tableofcontents
	\newpage
	\maintext
	
	\section{Manifolds and Tensors}
	\lecnr{1}
	Motivation: We study Differential Geometry because in General Relativity, there will not be a preferred coordinate system.

	\subsection{Introduction}

	\begin{ex}
		Unit sphere $S^2$ in $\mathbb{R}^3$: choose spherical polar coordinates, we find that they are not well-defined at poles $\theta = 0, \pi$ as $\phi$ is discontinuous at $\phi = 0$ or $2\pi$.
	\end{ex}

	\subsection{Differentiable Manifolds}

	\begin{defi}
		An $n$-dimensional \emph{differentiable manifold} is a set $M$ together with a collection of subsets $\mathcal{O}_{\alpha}$ such that
		\begin{enumerate}
			\item $\bigcup_{\alpha} \mathcal{O}_\alpha = M$, i.e.\ the subsets $\mathcal{O}_{\alpha}$ cover $M$;
			\item For each $\alpha$ there is a one-to-one and onto map $\phi_{\alpha}: \mathcal{O}_{\alpha} \to \mathcal{U}_{\alpha}$ where $\mathcal{U}_{\alpha}$ is an open subset of $\mathbb{R}^n$;
			\item If $\mathcal{O}_{\alpha}$ and $\mathcal{O}_{\beta}$ overlap, i.e.\ $\mathcal{O}_\alpha \cap \mathcal{O}_\beta \neq \varnothing$ then $\phi_{\beta} \cpf \phi_{\alpha}^{-1}$ maps from $\phi_{\alpha}(\mathcal{O}_{\alpha} \cap \mathcal{O}_{\beta}) \subset \mathcal{U}_{\alpha} \subset \mathbb{R}^n$ to $\phi_{\beta}(\mathcal{O}_\alpha \cap \mathcal{O}_\beta) \subset \mathcal{U}_\beta \subset \mathbb{R}^n$. We require this map to be smooth (infinitely differentiable).    
		\end{enumerate}
		
		The maps $\phi_\alpha$ are called \emph{charts} or \emph{coordinate systems}. The set $\{\phi_\alpha\}$ is called an \emph{atlas}.
	
	\end{defi}

	Sometimes write $\phi_\alpha(p) = (x_\alpha^1 (p), x_\alpha^2 (p), \cdots, x_\alpha^n (p))$ and refer to $x^i_\alpha(p)$ as the coordinates of $p$.

	Define a topology on $M: \mathcal{R} \subset M$ is \emph{open} iff $\phi_\alpha (\mathcal{R}\cap \mathcal{O}_\alpha) \subset \mathcal{U}_{\alpha}$ is open $\forall \alpha$.  

	\begin{ex}
		$\mathbb{R}^n$: single chart $\phi: (x^1, \dots, x^n) \to (x^1, \dots, x^n)$. 
	\end{ex}

	\begin{ex}
		$S^1 = \{(\cos \theta, \sin \theta): \theta \in \mathbb{R}\}$ is a 1d differentiable manifold. $\theta\in [0,2\pi)$ is not a good chart because $[0,2\pi)$ is not open. Instead, we use two charts. 
		
		Let $\mathcal{O}_1 = S^1 - \{P\}$ and $\mathcal{U}_1 = (0,2\pi)$. Define: $\phi_1: \mathcal{O}_1 \to \mathcal{U}_1$ by $\phi_1(p) = \theta_1$. 
		
		Let $\mathcal{O}_2 = S^1 - \{Q\}$ and $\mathcal{U}_2 = (-\pi,\pi)$. Define: $\phi_2: \mathcal{O}_2 \to \mathcal{U}_2$ by $\phi_2(p) = \theta_2$. 
		
		$\mathcal{O}_1 \cap \mathcal{O}_2$ is union of ``upper'' and ``lower'' semi-circles. On ``upper'' semi-circle, $\phi_2 \cpf \phi_1^{-1}(\theta_1) = \theta_2 = \theta_1$ is smooth. On ``lower'' semi-circle, $\phi_2 \cpf \phi_1^{-1}(\theta_1) = \theta_2 = \theta_1 - 2\pi$ is also smooth.
	\end{ex}

	\begin{ex}
		$S^2 = \{(x,y,z)\in \mathbb{R}^3 : x^2+y^2+z^2 = 1\}$ is a 2d differentiable manifold. Consider the conventional coordinate transformation $$x = \sin \theta \cos \phi,\qquad y = \sin \theta \sin \phi,\qquad z = \cos\theta$$ which is defined on $\theta \in [0,\pi]$ and $\phi \in [0,2\pi)$. The map $(x,y,z) \mapsto (\theta, \phi)$ is not a chart because $[0,2\pi]\times [0,2\pi)$ is not an open set.
		
		Let $\mathcal{O}$ be $S^2$ with points with $\theta = 0, \pi$ or $\phi = 0$ deleted.

		Define a chart $\psi: \mathcal{O} \to \mathcal{U} = (0,2\pi) \times (0,2\pi)$ by $\psi: (x,y,z) \mapsto (\theta,\phi)$
		
		Define $(\theta', \phi')$ by
		\[
			x = -\sin \theta' \cos \phi', \qquad y = \cos \theta', \qquad z = \sin\theta'\sin\phi' 
		\]

		Define $\psi' : \mathcal{O}' \to \mathcal{U}' = (0,\pi)\times (0,2\pi)$ by $\psi' : (x,y,z) \mapsto (\theta',\phi')$ 
		
		$S^2  = \mathcal{O} \cup \mathcal{O}'$

		$\psi \circ \psi'^{-1}$ and $\psi' \cpf \psi^{-1}$ are smooth on $\mathcal{O} \cap \mathcal{O}' \Rightarrow \{\psi,\psi'\}$ is atlas for $S^2$. 
	\end{ex}

	\begin{ex}[Product manifold]
		Cartesian product of sets $X$ and $Y$ is
		\[
			X \times Y = \{(x,y): x\in X, y\in Y\}
		\]
		
		Given differential manifolds $M$ and $M'$ of dimensions $n$ and $n'$ we can make $M \times M'$ into a differentiable manifold of dimension $n + n'$:

		charts $\phi_\alpha: '\mathcal{O}_\alpha \to \mathcal{U}_\alpha \subset \mathbb{R}^n$ and $\phi_\alpha' : \mathcal{O}_\alpha' \to \mathcal{U}_\alpha' \subset \mathbb{R}^{n'}$ fir $M,M'$ 

		Let $\phi_{\alpha\beta}: \mathcal{O}_\alpha \times \mathcal{O}_\beta' \to \mathcal{U}_\alpha \times \mathcal{U}_\beta' \subset \mathbb{R}^{n + n'}, \phi_{\alpha\beta}(p,q) = (\phi_\alpha(p),\phi_\beta'(q))$

		$\{\phi_{\alpha\beta}\}$ is an atlas for $M\times M'$ 

		Generalizes to multiple factors e.g.\ we can define the $n$-torus as $T^n = S^1 \times S^1 \times \cdots \times S^1$ 
	\end{ex}

	\begin{defi}
		Two atlases are \emph{compatible} if their union is also an atlas. The union of all atlases compatible with a given atlas is a \emph{complete} atlas. (This is not contained in any other atlas.)
	\end{defi}

	We'll always assume we are dealing with a complete atlas (unlike most of our examples so far!)

	\subsection{Smooth Functions}
	Function $f: M \to \mathbb{R}$

	Chart $\phi: \mathcal{O} \to \mathcal{U} \subset \mathbb{R}^n$
	
	$F \equiv f \cpf \phi^{-1}: \mathcal{U}\to \mathbb{R}$
	
	\begin{defi}
		$f$ is \emph{smooth} iff $F$ is smooth for all charts $\phi$.
	\end{defi}

	\begin{ex}
		On $S^1 = \{(x,.y) \in \mathbb{R}^2 : x^2+y^2 = 1\}$ let $f(x,y) = x$ 

		$f \cpf \phi_1^{-1}(\theta_1) = f(\cos \theta_1 , \sin \theta_1) = \cos \theta_1$ is smooth.
		
		$f \cpf \phi_2^{-1}(\theta_2) = f(\cos \theta_2 , \sin \theta_2) = \cos \theta_2$ is smooth.

		Let $\phi$ be any other chart: $f \cpf \phi^{-1} = (f \cpf \phi_i^{-1}) \cpf (\phi_i \cpf \phi^{-1})$ smooth.
	\end{ex}

	\begin{ex}
		Any $M$ with a chart $\phi: \mathcal{O} \to \mathcal{U} \subset \mathbb{R}^n$,
		$$\phi: p \mapsto (x^1(p),x^2(p), \cdots, x^n(p))$$

		Regard $x^1$ as a function on $\mathcal{O} \subset M$.

		For any other chart $\phi', x^1 \cpf \phi'^{-1}$ is smooth because it is the first component of the map $ \phi \cpf \phi'^{-1}$ (smooth by the definition of a manifold).
	\end{ex}

	\begin{ex}
		Usually we specify $F$ instead of $f$.

		Given atlas $\{\phi_\alpha\}$, specify $F_\alpha: \mathcal{U}_\alpha \to \mathbb{R}$ and define $f = F_\alpha \cpf \phi_\alpha$.

		Must check result is independent of $\alpha$ on chart overlaps.

		e.g.\ on $S^1$ define
		\[
			F_1 : (0,2\pi) \to \mathbb{R} \qquad \theta_1 \mapsto \sin(m \theta_1)
		\]
		\[
			F_2 : (-\pi,\pi) \to \mathbb{R} \qquad \theta_2 \mapsto \sin(m\theta_2)
		\]
		with $ m \in \mathbb{Z}$.

		On the chart overlaps $F_1 \cpf \phi_1 = F_2 \cpf \phi_2$ because $\theta_2 = \theta_1$ or $\theta_1 - 2\pi$.
	\end{ex}
\end{document}