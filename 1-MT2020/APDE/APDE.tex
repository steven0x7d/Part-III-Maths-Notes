\documentclass[a4paper,11pt]{article}

\usepackage{../../zyancamlec}

\def\ntripos{Mathematical Tripos}
\def\npart{III}
\def\ncourse{Analysis of Partial Differential Equations}
\def\nscourse{AnaPDE}
\def\nlecturer{C.\ Warnick}
\def\nterm{Michaelmas}
\def\nyear{2020}

\begin{document}
	\maketitlepage
	\preliminaries
	\section*{Introduction}
	This course serves as an introduction to the mathematical study of Partial Differential Equations (PDEs). The theory of PDEs is nowadays a huge area of active research, and it goes back to the very birth of mathematical analysis in the 18th and 19th centuries. The subject lies at the crossroads of physics and many areas of pure and applied mathematics.

	\noindent The course will mostly focus on developing the theory and methods of the modern approach to PDE theory. Emphasis will be given to functional analytic techniques, relying on a priori estimates rather than explicit solutions. The course will primarily focus on approaches to linear elliptic and evolutionary problems through energy estimates, with the prototypical examples being Laplace’s equation and the heat, wave and Schr\"odinger equations.

	\noindent The following concepts will be studied: well-posedness; the Cauchy problem for general (non-linear) PDE; Sobolev spaces; elliptic boundary value problems: solvability and regularity; evolutionary problems: hyperbolic, parabolic and dispersive PDE.

	\section*{Pre-requisites}
	There are no specific pre-requisites beyond a standard undergraduate analysis background, in particular a familiarity with measure theory and integration. The course will be mostly self-contained and can be used as a first introductory course in PDEs for students wishing to continue with some specialised PDE Part III courses in the Lent and Easter terms.
	\newpage
	\tableofcontents
	\newpage
	\maintext
	Hi
	

\end{document}